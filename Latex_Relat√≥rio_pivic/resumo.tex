\begin{huge}
\textbf{Resumo}
\\

\end{huge}

\noindent
	
	No presente trabalho procura-se desenvolver dinamicamente e termicamente um canal turbulento de Poiseuille unidimensional por meio de uma metodologia semi-analítica. Para tal,  utilizou-se as equações diferenciais da energia térmica, de Navier-Stokes e da continuidade, que foram desenvolvidas numericamente e acopladas em um código final capaz de simular o sistema em questão. Foi analisado o erro com base em DNS (Simulações Numéricas Diretas) a fim de se validar o método. \\
	A partir dos resultados iniciais foi notado que algumas constantes tinham grande influencia sobre o resultado (Prandtl turbulento e constante de cebeci), e se experimentou muda-las de forma a se obter o melhor resultado, com isso criou-se modelos simplificados para os mesmos de forma a se melhorar a acurácia do método.

\newpage

\begin{huge}
	\textbf{Abstract}
	\\
	
\end{huge}

\noindent

In the present work, it was developed, dynamically and thermally, a one-dimensional Poiseuille turbulent channel flow by means of a semi-analytical methodology. For that, we used the differential equations of thermal energy, Navier-Stokes and continuity, which were developed numerically and coupled in a final algorithm capable of simulating the channel. The error based on DNS (Direct Numerical Simulations) was analyzed in order to validate the method. \\
From the initial results it was noticed that some constants had great influence on the results (turbulent Prandtl number and Cebeci number), and if it was tried to change them in order to obtain the best result, with that simplified models were created for them in order to improve the accuracy of the method.

\newpage