% LaTeX .tex
% Example for the proceedings of the  25th International Congress of Mechanical Engineering
% COBEM 2019
% October, 20-25, 2019, Uberlândia, MG, Brazil
% Based on the template of the proceedings of COBEM2015 and COBEM2017

\documentclass[10pt,fleqn,a4paper,twoside]{article}
\usepackage{abcm}
\def\shortauthor{F. Author, S. Author and T. Author (update this heading accordingly)}
\def\shorttitle{Paper Short Title (First Letters Uppercase, make sure it fits in one line)}
\usepackage{blindtext}
\begin{document}
\fphead
\hspace*{-2.5mm}\begin{tabular}{||p{\textwidth}}
\begin{center}
\vspace{-4mm}
\title{Development and certification of an winding machine}
\end{center}
\authors{Felipe Jos\'{e} Oliveira Riberio} \\
\authors{Izaaque Aniceto Macedo} \\
\authors{Rodrigo Lira Reis Neves} \\
\authors{Alexandre Zuquete Guarato} \\
\institution{Federal University of Uberlândia (UFU), Av. João Naves de Ávila, 2121, Campos Santa Mônica, Uberlândia, MG } \\
\institution{feliperibeiro.ufu@gmail.com} \\
\institution{izaaque@live.com} \\
\institution{rodrigolira1999@gmail.com} \\
\institution{azguarato@ufu.br} \\
\\
\abstract{\textbf{Abstract.} \blindtext .}\\
\\
\keywords{\textbf{Keywords:} winding, composite materials, machinery construction \dots{}}\\
\end{tabular}

\section{INTRODUCTION}

Temos na indústria um aumento na utilização de matérias compostos, “A utilização de matérias compósitos na indústria deve-se sobretudo ás elevadas performances que se conseguem obter a partir dos seus materiais constituintes, bem como dos preços competitivos na obtenção do produto final, em suma de toda a sua cadeia de produção. ” [1]
Mas para estes produtos serem um diferencial para aplicação na indústria, será necessário que seu sistema de produção tenha um performasse optimizada e automatizada para que a produção seja rentabilizada. Podemos observar que para maximizar a produção, utilizamos ferramentas como CAD/CAM para gerar modelos que possam apresentar dados próximos aos reais.
Logo temos a técnica do enrolamento filamentar para a construção de matérias compostos.
“O enrolamento filamentar é uma técnica de fabrico para formação de plásticos reforçados, com elevados índices resistência/peso e rigidez/peso. Isto é possível pela exploração das propriedades de resistência e rigidez das fibras contínuas envolvidas numa matriz de resina orgânica ou inorgânica.”[2]
Está técnica de enrolamento filamentar utiliza fibras continuas em diferentes direções, na qual utiliza um mandril em rotação, que será a base da estrutura a ser fabricada. Este mandril está rotacionando enquanto são adicionadas as fibras e a sua matriz, após ser feita a cura, esse mandril pode ser retirado ou fazer parte da estrutura desta nova peça.
“O enrolamento filamentar é executado em máquinas automáticas especiais adequadas a esse tipo de técnica. O controlo preciso do enrolamento e a direção das fibras são requisitos necessários para obtenção de boas performances.”[2]
Portanto temos que ter o conhecimento teórico dos aspectos que compõem esse tipo de máquina de enrolamento filamentar para ser possível o desenvolvimento de uma nova máquina que tenha o mandril sendo feito por meio de uma impressão 3D acoplado ao enrolamento filamentar em uma bancada fixa projetada para alcançar o objetivo de um tubo de secção fina, sendo composto da estrutura interna de um tubo feito pela impressão 3D com o composto de fibra e matriz acoplado a este. 
Gerando uma estrutura com características únicas para aplicação na engenharia um tudo de parede fina com material composto e impressão 3D, sendo realizado testes para determinar as propriedades da nova estrutura, gerada pela bancada fixa de enrolamento filamentar para tubos de paredes finas desenvolvida neste processo.   

\section{WINDING MACHINE MANUFACTURING}

----> explicar objetivo da máquina e justificar dimensões da máquina. (lembrar de 1 metro de area util)

\subsection{MECHANICAL PARTS}
----> Listar peças produzidas e para quê cada uma serve.

\subsection{ELECTRONIC PARTS}

----> explicar aparato eletrônico. (um arduino UNO, dois motores de passo e drives e fonte e botão)

\section{MECHANISM VALIDATION}
Para a validação da bancada, foram criados tubos de parede fina feitos integralmente de compósitos, para isso, .







\section{ACKNOWLEDGEMENTS}
The authors would like to thank the following professors and institutions: Dr. Ruham Pablo Reis, Lmest (Structural Mechanics Laboratory), FEMEC (Mechanical engineering College), UFU (Federal University of Uberl\^andia) and, especially, EPTA(Propulsion and Aerospace Technology Team) for financial support.

\section{REFERENCES} 

\bibliographystyle{abcm}
\renewcommand{\refname}{}
\bibliography{bibfile}

\end{document}
