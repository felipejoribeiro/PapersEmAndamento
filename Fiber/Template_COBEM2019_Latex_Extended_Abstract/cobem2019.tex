% LaTeX .tex
% Example for the proceedings of the  25th International Congress of Mechanical Engineering
% COBEM 2019
% October, 20-25, 2019, Uberlândia, MG, Brazil
% Based on the template of the proceedings of COBEM2015 and COBEM2017

\documentclass[10pt,fleqn,a4paper,twoside]{article}
\usepackage{abcm}
\def\shortauthor{F. Author, S. Author and T. Author (update this heading accordingly)}
\def\shorttitle{Paper Short Title (First Letters Uppercase, make sure it fits in one line)}
\usepackage{blindtext}
\begin{document}
\fphead
\hspace*{-2.5mm}\begin{tabular}{||p{\textwidth}}
\begin{center}
\vspace{-4mm}
\title{Development and certification of an winding machine}
\end{center}
\authors{Felipe Jos\'{e} Oliveira Riberio} \\
\authors{Izaaque Aniceto Macedo} \\
\authors{Rodrigo Lira Reis Neves} \\
\authors{Alexandre Zuquete Guarato} \\
\institution{Federal University of Uberlândia (UFU), Av. João Naves de Ávila, 2121, Campos Santa Mônica, Uberlândia, MG } \\
\institution{feliperibeiro.ufu@gmail.com} \\
\institution{izaaque@live.com} \\
\institution{rodrigolira1999@gmail.com} \\
\institution{azguarato@ufu.br} \\
\\
\abstract{\textbf{Abstract.} \blindtext .}\\
\\
\keywords{\textbf{Keywords:} winding, composite materials, machinery construction \dots{}}\\
\end{tabular}

\section{INTRODUCTION}

----> Breve descrição da equipe epta

----> motivação 

----> Introduzir no processo de filamentação 

----> breve descrição dos processos de montagem e certificação (não precisa certificação, felipe complementa essa parte)

\section{WINDING MACHINE MANUFACTURING}

----> explicar objetivo da máquina e justificar dimensões da máquina. (lembrar de 1 metro de area util)

\subsection{MECHANICAL PARTS}
----> Listar peças produzidas e para quê cada uma serve.

\subsection{ELECTRONIC PARTS}

----> explicar aparato eletrônico. (um arduino UNO, dois motores de passo e drives e fonte e botão)

\section{MECHANISM VALIDATION}
Para a validação da bancada, foram criados dois tipos de corpos de prova, sendo estes, corpos de prova controle, cujo material de fabricação consistia meramente de material impresso em 3D, e também corpos de prova filamentados pela bancada.







\section{ACKNOWLEDGEMENTS}
The authors would like to thank the following professors and institutions: Dr. Ruham Pablo Reis, Lmest (Structural Mechanics Laboratory), FEMEC (Mechanical engineering College), UFU (Federal University of Uberl\^andia) and, especially, EPTA(Propulsion and Aerospace Technology Team) for financial support.

\section{REFERENCES} 

\bibliographystyle{abcm}
\renewcommand{\refname}{}
\bibliography{bibfile}

\end{document}
