\chapter{Introdução}\label{chap_intro}

%\pagenumbering{arabic} % numerar as paginas em algarismos arabicos (ou romanos)
\thispagestyle{empty} %Oculta o numero da primeira pagina do capitulo
\vspace{3ex}

A dinâmica dos fluidos e das estruturas imersas são ramos da Engenharia Mecânica com as quais se estuda o comportamento das interações fluido-estruturais, visando aperfeiçoar vários processos na indústria, assim como entender fenômenos na natureza. O estudo desse tipo de fenômeno pode ser feito de duas maneiras: a experimentação material e a experimentação virtual. Em ambas as frentes é necessária a modelagem física do problema em análise e a montagem de bancadas experimentais que representem a física do problema a ser estudado. Para a modelagem material, faz-se necessária a construção de bancadas e as respectivas instrumentações para permitir a coleta de dados e análise estatística dos resultados obtidos. A modelagem física consiste na avaliação do problema de interesse e determinação de suposições físicas que visam viabilizar a análise. Na segunda maneira, é necessária a modelagem física e matemática do problema de interesse. A modelagem matemática consiste na obtenção de equações diferenciais, integrais e/ou integro-diferenciais que modelam a física assciada e depois utilizar métodos numéricos apropriados para a discretização das equações. 

Os métodos de análise computacional estão em franco crescimento. É importante destacar que o método computacional não substitui o experimental material, mas o complementa. Além disso, esta metodologia é muito versátil, permitindo uma análise minuciosa do problema físico e uma maior flexibilidade em relação às condições físicas. Alguns experimentos podem ser perigosos de se reproduzir em laboratório, ou até mesmo impossíveis. Algumas desvantagens são a necessidade de modelos numéricos adequados e de computadores robustos, conforme o problema que se está analisando.

A interação entre escoamentos e estruturas é um problema complexo e recorrente em aplicações de engenharia. Esse fenômeno pode ser encontrado em aeronaves, motores a jato, dutos, reatores nucleares e químicos, pontes, torres, plataformas \textit{off-shore}, válvulas de compressores, coração, aneurismas, entre outros. A dinâmica dos fluidos computacional, aliada à solução numérica das equações que modelam a movimentação de estruturas, é uma grande aliada na compreensão dos problemas de interação fluido-estrutura. Trata-se de um problema multidisciplinar, visto que envolve, por exemplo, a mecânica dos fluidos, mecânica das estruturas, engenharia de software e a ciência da computação.

Os escoamentos sobre estruturas cilíndricas podem ser a fonte de vibrações induzidas por estruturas turbilhonares. Essas vibrações podem induzir um aumento das forças fluidodinâmicas, ou seja, arrasto e sustentação, levando assim a um aumento dos esforços aplicados sobre as estruturas. Alguns resultados indicam que escoamentos bidimensionais sobre cilindros circulares mudam, por exemplo, o coeficiente de arrasto médio de 1,3 para 2,2. Além disso, as vibrações podem causar nucleação e propagação de trincas na estrutura conduzindo-a a falha em virtude da fadiga. Em alguns casos, o valor RMS do coeficiente de sustentação pode ser alterado de 0,3 para 1,75, dependendo do regime de operação \cite{Chern2014}. Esses são resultados que justificam a preocupação com o processo de interação fluido-estrutura em cilindros. Isso é especialmente importante quando esses cilindros são dutos pelos quais petróleo ou gás natural são transportados, sobre os quais se têm ondas e/ou correntes marítimas atuando. A manutenção desse tipo de duto é cara, visto que podem estar a centenas de metros da superfície. Qualquer falha nessas estruturas pode causar desastres ambientais e grandes prejuízos. Por isso é importante entender como o processo de interação fluido-estrutura atua sobre a dinâmica do duto para prevenir falhas.

A pesquisa apresentada neste projeto é resultado de cooperação entre o Laboratório de Mecânica dos Fluidos (MFLab), o Laboratório de Mecânica de Estruturas Prof. José Eduardo Tannús Reis (LMEst) da Universidade Federal de Uberlândia (UFU) e com o centro de pesquisa (CENPES) da Petróleo Brasileiro S.A. (Petrobras). A pesquisa está sendo feita utilizando a ferramenta computacional MFSim, que está sendo desenvolvido no MFLab. Nessa ferramenta é possível a execução de simulações de escoamentos incompressíveis levando em consideração a movimentação de estruturas. Para isso, o método da Fronteira Imersa é utilizado. Esta metodologia é particularmente adequada para os problemas que envolvem interação fluido-estrutura, pois permite tratar os domínios do fluido e da estrutura de forma independente. As equações que modelam o escoamento são resolvidas em um domínio euleriano fixo e cartesiano, enquanto a superfície do corpo imerso (estrutura) é representada por um conjunto de pontos lagrangeanos. Através dessa técnica, as forças na interface entre a estrutura e o fluido são avaliadas e utilizadas tanto nos códigos associados ao fluido para imposição da condição de contorno de não deslizamento, quanto na rotina estrutural para o cálculo dos deslocamentos e velocidades da estrutura.

A motivação para a realização deste projeto de pesquisa se dá pelo fato que é alto o custo computacional de simulações associadas ao dimensionamento de estruturas imersas. Este problema se torna maior caso a aplicação de processos de otimização seja necessária. Assim, neste projeto é desenvolvida uma ferramenta computacional dedicada à construção de metamodelos dessas estruturas. Ao final do projeto, a análise dinâmica de diferentes estruturas imersas poderá ser realizada com baixo custo computacional. Neste caso, em vez de resolver as equações do movimento associadas ao problema de interação fluido-estrutura, os usuários da ferramenta acessarão um mapa criado com funções e polinômios obtidos a partir de simulações de alto custo realizadas previamente. Desta forma, reduzindo o tempo computacional para alguns segundos de cálculo viabilizando a utilização de mais testes e avaliações no projeto estrutural. 

Neste projeto, serão realizadas atividades voltadas para o desenvolvimento de uma ferramenta computacional usada na construção de metamodelos para análise estrutural. Assim, diferentes análises e procedimentos de otimização poderão ser realizados de forma mais rápida e eficiente no projeto deste tipo de estruturas. Além disso, será realizada uma análise computacional de escoamentos turbulentos tridimensionais sobre um duto com \textit{strakes}. Desta forma, ao final do projeto, a análise de diferentes estruturas submersas será realizada com baixo custo computacional. Neste sentido, será utilizada uma plataforma desenvolvida no MFlab que vem sendo empregada em aplicações de interesse da indústria de óleo e gás. Novos desenvolvimentos nesse código serão necessários. Com esses novos desenvolvimentos a ferramenta será potencializada para a aplicação em pauta. Esta é a importância deste projeto para o setor de petróleo, gás natural, energia e biocombustíveis.

A presente proposta tem por objetivo dar sequência à cooperação com a construção da ferramenta de metamodelagem para estruturas de \textit{risers}. Além disso, será realizada a simulação numérica de escoamentos tridimensionais sobre dutos com \textit{strakes}. As dimensões dos \textit{strakes}, ou seja, a altura e passo da hélice, são parâmetros críticos no projeto destes sistemas. Assim sendo, torna-se interessante avaliar sua eficiência através de simulações computacionais para diferentes condições de operação. São objetivos específicos: 1) Metamodelagem estrutural: a parametrização do metamodelo é dada pelo conjunto de dados estruturais e hidrodinâmicos, mas a metamodelagem constitui apenas a análise estrutural. O carregamento hidrodinâmico é dado de entrada; 2) Geração de sinais de CD e CL através de CFD pelo MFSim sobre os modelos que temos de cilindros fixos com strakes, ensaiados em canal de água corrente: análises de CFD puras; 3) Metamodelagem para levantamento de cargas sobre cilindros com \textit{strakes} sem acoplamento estrutural (metamodelagem apenas de CFD).

Neste contexto, o presente relatório parcial é organizado da seguinte maneira. \textcolor{red}{\textbf{CONCLUIR CONCLUIR CONCLUIR CONCLUIR CONCLUIR CONCLUIR CONCLUIR}}.