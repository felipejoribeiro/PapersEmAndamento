\chapter{Revisão bibliográfica} \label{chap_revisao}


Apesar do avanço no desempenho dos computadores, um elevado custo computacional pode ser obtido ao se trabalhar com a solução de alguns problemas de engenharia, como é o caso da interação fluido-estrutura. Isto se deve ao fato que modelos mais representativos e, consequentemente, mais custosos computacionalmente, foram sendo desenvolvidos ao longo dos últimos anos. Neste sentido, uma solução que vem sendo aplicada visando a redução do custo computacional é a utilização dos chamados modelos substitutos, também denominados modelos reduzidos ou metamodelos {\it (surrogate models)}. 

Os metamodelos são baseados em funções de interpolação ajustadas de forma a representar os modelos originais \cite{VANBEERSKLEIJNEN2004}. Segundo \citeonline{MENDONCA2016} {\it apud} \citeonline{Simpson2001}, um metamodelo é utilizado para substituir a solução de um modelo que demanda simulações complexas. Assim, um modelo substituto deve retratar fielmente o comportamento do sistema original com um tempo computacional menor \cite{Luo2014}. Metamodelos podem, inclusive, ser aplicados em problemas onde a convergência do procedimento iterativo é lenta \cite{VILLAR2016}.

Dentre os métodos de metamodelagem disponíveis na literatura, destacam-se a metodologia da Superfície de Resposta (\textit{Response Surface Methodology} - RSM)  \cite{PINA2010}, Função de Base Radial (\textit{Radial Basis Function} - RBF) \cite{Rueda2019,YIN2018}, Inverso da Distância Ponderada (\textit{Inverse Distance Weighting} - IDW) \cite{HWANG2018}, as Redes Neurais Artificiais (\textit{Artificial Neural Networks} - ANNs) \cite{NAZERIAN2018,SANTOS2013}, {\it Splines} de Regressão Adaptativa Multivariada (\textit{Multivariate Adaptive Regression Splines} – MARS) \cite {LIRAJr2012,friedman1991} e o Método {\it Kriging} (\textit{Kriging Method}) \cite{Erickson2018,KLEIJNEN2015}, também conhecido como Processo Gaussiano. 

A RSM é uma das técnicas mais populares da metamodelagem que tem por objetivo substituir uma resposta complexa por um polinômio. No método RSM, em geral, usa-se polinômios de primeira ou segunda ordem, pois para polinômios de ordem maior existe a dificuldade na amostragem de dados para que se consiga aproximar todos os coeficientes do polinômio. Segundo \citeonline{VarunAryaPrakash2018}, são inúmeros os pesquisadores que sugerem a utilização da otimização em conjunto com a abordagem RSM. \citeonline{VenterHaftkaStarnes1996} utilizaram um metamodelo RSM para aproximar o fator de concentração de tensão e a carga de flambagem para uma placa isotrópica homogênea com alteração na espessura. \citeonline{Gogu2013} combinou o RSM com a modelagem de base reduzida para representar campos térmicos. Neste caso, o método proposto se mostrou mais eficiente que o RSM tradicional para representar o fenômeno associado. 

Visto que a RSM se utiliza de modelos polinomiais de ordem menor que três, em alguns casos esta aproximação é insuficiente para modelar com precisão funções não lineares. Para a consideração de polinômios de ordem maior há a inconveniência de estimar todos os coeficientes do polinômio, assim, surgiu a necessidade de abordar outros metamodelos. 

O método de metamodelagem RBF surgiu com \citeonline{hard1971}, desenvolvido para a interpolação multivariada em aplicações de cartografia. Segundo \citeonline{QUEIPO2005}, a interpolação pelo RBF utiliza uma combinação de funções consideradas funções de base. Estas funções são simétricas e centralizadas em cada ponto de amostragem, possuindo a característica de que a sua resposta diminui ou aumenta monotonamente com a distância ao ponto central. Como os demais métodos de metamodelagem, o RBF é aplicado em diversos problemas. 

\citeonline{ZHOU2013} utilizaram a abordagem RBF para modelar uma estrutura de uma ponte estaiada, em que foram investigadas diferentes funções de base, como multicamadas, gaussiana, quadrática inversa, entre outras. Como conclusão, os autores verificaram que o método tem alta precisão de aproximação. {\color{red}Em \citeonline{sun2011radial}}, este metamodelo foi usado para modelar critérios de fratura e enrugamento no projeto de formação de chapas metálicas obtendo uma boa aproximação.

As ANNs pertencem a outra classe de metamodelagem, sendo vastamente aplicadas. As ANNs se assemelham ao cérebro humano, sendo que conhecimento ocorre por meio de um processo de aprendizagem e pesos sinápticos são utilizados para armazenar o conhecimento adquirido \cite{MENDONCA2016}. As ANNs são capazes de aproximar funções, reconhecer e controlar padrões. Podem adquirir conhecimento por experiência e tem como principais características o aprendizado por meio de exemplos, possuem alto grau de tolerância a falhas, adaptabilidade e habilidade de aprendizagem \cite{VILLAR2016}.

Em \citeonline{MENDONCA2016}, ANNs foram usadas para analisar o comportamento de placas laminadas de material compósito reforçado por fibras. Neste caso, o cálculo da segurança à flambagem das placas foram calculadas durante um processo de otimização. \citeonline{NAZERIAN2018} utilizaram a RSM e as ANNs para modelar a resistência {\color{red} à} flexão de painéis de fibra de gesso. As métricas de precisão utilizadas para comparar os metamodelos foram o coeficiente de determinação, erro médio quadrático e o erro absoluto médio, nos quais as ANNs se mostraram mais precisas.

Outra ferramenta de metamodelagem que vem se mostrando bastante eficiente é o {\it Kriging}, uma vez que possui inúmeras funções de correlação que podem ser escolhidas para construir a aproximação \cite{Simpson2001}. Este metamodelo interpola as respostas nos pontos iniciais e consegue prever as respostas para novos pontos, permitindo previsões espaciais por interpolação linear de pontos observados com mínima variância \cite{CARVALHO2017}. Inicialmente, o {\it Kriging} foi empregado para a mineração de ouro e geologia por Daniel G. Krige \cite{Krige1951}. Porém, este método foi formalizado pelo matemático francês George Matheron \cite{Matheron1963}. 

Segundo \citeonline{MARINONI2003} {\it apud} \citeonline{olea1991}, o metamodelo {\it Kriging} é uma coleção de técnicas de regressão linear que leva em conta a dependência estocástica entre os dados. Ainda de acordo com \citeonline{VILLAR2016}, esta abordagem pode ser definida como uma família de algoritmos de regressão que tem por objetivo a minimização da variância. 

\citeonline{SAKATA2003} aplicaram o Kriging na otimização estrutural de um cilindro enrijecido para um problema de alta frequência. \citeonline{Srivastava2004} aplicaram os metamodelos {\it Kriging} e RSM em um problema de aeronaves de transporte civil de alta velocidade. Por se tratar de um problema complexo de otimização, que tem como objetivo a minimização do peso bruto de decolagem, existem várias variáveis de projeto. Assim, o trabalho busca aproximar a função objetivo utilizando os métodos de metamodelagem mencionados. Os autores verificaram que o {\it Kriging} foi mais preciso em comparação {\color{red} à} abordagem RSM. 

O {\it Kriging} divide-se em duas partes: uma função de tendência da média e um processo gaussiano. Na literatura existem inúmeras variantes do metamodelo {\it Kriging}, como o {\it Kriging} comum ou normal (\textit{Ordinary Kriging}), que considera a média constante em todo o domínio, e o {\it Kriging} universal (\textit{Universal Kriging}), onde determina-se a média utilizando uma função polinomial de primeira ou segunda ordem  {\color{red}\cite{lataniotis2015uqlab}}. De acordo com \citeonline{HWANG2018}, o {\it Kriging} simples (\textit{Simple Kriging}) considera uma média constante como sendo conhecida. Já o {\it Kriging} comum considera a média constante como sendo desconhecida e, portanto, esta deve ser estimada. 

O {\it Kriging} às cegas (\textit{Blind Kriging}) é abordado em \citeonline{ULAGANATHAN2015}. Neste caso, o conjunto de funções de tendência é estimado a fim de melhorar a capacidade de aproximação do {\it Kriging}. 

\citeonline{ROCHA2012} estudou outra vertente do método {\it Kriging} conhecida como \textit{Co-Kriging}. Este método estima uma variável de interesse (variável de alta fidelidade) utilizando suas próprias informações e de outras variáveis (de baixa fidelidade). Neste caso, a amostragem da variável de alta fidelidade possui um alto custo computacional e, consistentemente, possui uma amostragem de dados insuficiente. Já a amostragem das variáveis de baixa fidelidade é pouco custosa computacionalmente. Portanto, o método correlaciona as informações das variáveis de alta fidelidade e de baixa fidelidade para estimar as respostas com maior precisão.

Para \citeonline{righetto2013}, o \textit{Co-Kriging} é satisfatório quando as variáveis apresentam um grau de correlação significativo. Destaca-se que a estimativa da variável de interesse é determinada em função das variáveis secundárias, utilizando covariâncias simples e cruzadas. Os métodos {\it Kriging} e \textit{Co-Kriging} foram comparados em um problema de condutividade térmica por \citeonline{KNOTTERS1995}. Como conclusão, os autores mencionam que o {\it Kriging} produziu bons resultados.

Os metamodelos {\it Kriging} indicador (\textit{Indicator Kriging}), o {\it Kriging} probabilístico (\textit{Probabilistic Kriging}) e o {\it Kriging} disjuntivo (\textit{Disjunctive Kriging}) também são encontrados na literatura. 

O {\it Kriging} indicador é definido como sendo um processo de predição não linear, uma vez que requer uma transformação não linear na qual os dados são codificados em números binários se estiverem acima ou abaixo de um determinado valor de corte. O {\it Kriging} probabilístico é considerado como uma melhoria do {\it Kriging} indicador, no entanto, baseia-se no \textit{Co-Kriging} para obter resultados mais precisos. O método calcula a probabilidade dos valores estimados serem inferiores ou superiores a um determinado limite \cite{ASA2012}.

O {\it Kriging} disjuntivo é definido como sendo uma técnica não linear de estimativa. Nesse método, \textcolor{red}{consideram-se} dois parâmetros de entrada, um nível de corte e o nível de probabilidade crítica, sendo o {\it Kriging} indicador um caso especial deste método \cite{ahmed2001}. Vale ressaltar a aplicação dos métodos de metamodelagem nas mais diversas áreas, como exemplo, no trabalho de \citeonline{Dance2018}, em que os métodos {\it Kriging} comum e {\it Kriging} disjuntivo são utilizados como uma importante técnica na agricultura. 

\citeonline{Angelico2006} cita a aplicação do \textit{Co-Kriging} para estimar a variabilidade de solos, onde estima-se o pH e o teor de manganês de acordo com a matéria orgânica associada. O autor ainda cita várias outras contribuições científicas em que o mesmo método é aplicado. \citeonline{ULAGANATHAN2015} discute o \textit{DACE Toolbox} (Projeto e Análise de Experimentos Computacionais), que possui vários métodos {\it Kriging} implementados, a saber: {\it Kriging} simples, {\it Kriging} comum, {\it Kriging} universal e \textit{Co-Kriging}.

Em \citeonline{backlund2010}, os metamodelos {\it Kriging} e RBF foram avaliados em relação \textcolor{red}{à} capacidade de modelar funções não lineares. Os métodos foram comparados em relação ao seu custo computacional, precisão e número de amostras necessárias para sua formulação. Neste caso, o metamodelo RBF obtido se mostrou mais eficiente. \citeonline{silva2011} realizou o projeto de turbo máquinas utilizando a RBF.

\citeonline{CARVALHO2017}, aplicou os metamodelos {\it Kriging} comum, {\it Kriging} probabilístico e RBF para prever o desempenho da gasolina utilizada em um motor de combustão interna. Nesse caso, a RBF apresentou um melhor desempenho utilizando o coeficiente de determinação como métrica de avaliação.

Neste contexto, percebe-se a grande quantidade de metamodelos disponíveis na literatura e que suas aplicações em várias áreas da engenharia, geografia, entre outras. Desta forma, será abordado na sequência alguns dos metamodelos citados. É importante ressaltar que mais detalhes acerca do procedimento de metamodelagem serão apresentados, além de exemplos de casos de uso.
