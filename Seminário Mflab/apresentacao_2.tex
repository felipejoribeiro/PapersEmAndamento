\documentclass[xcolor=dvipsnames,10pt,aspectratio=169]{beamer}
%\documentclass[xcolor=dvipsnames,10pt]{beamer}
\usepackage{etex}
\usepackage{pgf,pgfarrows,pgfnodes,pgfautomata,pgfheaps,pgfshade}
\usepackage[absolute,overlay]{textpos} 
%\usepackage{algorithm}
\usepackage{amsmath,amssymb}
\usepackage[utf8]{inputenc} 
\usepackage{colortbl}
\usepackage{graphicx} 
\usepackage[brazil]{babel}
\usepackage{tabularx} 
\usepackage{multirow}
\usepackage{booktabs}
\usepackage{listings}
\usepackage{multimedia}
\usepackage{animate}
\usepackage{xcolor}
\usepackage{array}
\usepackage{longtable}
\usepackage{makecell}
\usepackage{caption}
\usetheme{Madrid} 

\lstset{ %
%	backgroundcolor=\color{white},   % choose the background color; you must add \usepackage{color} or \usepackage{xcolor}
%	basicstyle=\footnotesize,        % the size of the fonts that are used for the code
	basicstyle=\scriptsize,        % the size of the fonts that are used for the code
	breakatwhitespace=false,         % sets if automatic breaks should only happen at whitespace
	breaklines=true,                 % sets automatic line breaking
	captionpos=t,                    % sets the caption-position to bottom
	commentstyle=\color{mygreen},    % comment style
	deletekeywords={...},            % if you want to delete keywords from the given language
	escapeinside={\%*}{*)},          % if you want to add LaTeX within your code
	extendedchars=true,              % lets you use non-ASCII characters; for 8-bits encodings only, does not work with UTF-8
%	frame=single,                    % adds a frame around the code
	keepspaces=true,                 % keeps spaces in text, useful for keeping indentation of code (possibly needs columns=flexible)
	keywordstyle=\color{blue},       % keyword style
%	language=make,                 % the language of the code
	morekeywords={*,...},            % if you want to add more keywords to the set
%	numbers=left,                    % where to put the line-numbers; possible values are (none, left, right)
%	numbersep=5pt,                   % how far the line-numbers are from the code
	numberstyle=\tiny\color{mygray}, % the style that is used for the line-numbers
	rulecolor=\color{black},         % if not set, the frame-color may be changed on line-breaks within not-black text (e.g. comments (green here))
	showspaces=false,                % show spaces everywhere adding particular underscores; it overrides 'showstringspaces'
	showstringspaces=false,          % underline spaces within strings only
	showtabs=false,                  % show tabs within strings adding particular underscores
	stepnumber=2,                    % the step between two line-numbers. If it's 1, each line will be numbered
}

\definecolor{mygreen}{rgb}{0,0.6,0}
\definecolor{mygray}{rgb}{0.5,0.5,0.5}
\definecolor{mymauve}{rgb}{0.58,0,0.82}

\usecolortheme{beaver}
\newcommand{\ul}{\underline}
\setbeamertemplate{footline}{\scriptsize{\vspace*{0.3cm}\hspace*{15cm}\insertframenumber\,/\,\inserttotalframenumber}}
\setbeamertemplate{caption}[numbered]
\setbeamerfont{caption}{size=\fontsize{8}{5}}

\setbeamercolor{block title}{	bg=Sepia , fg = White}
\setbeamercolor{block body}{bg=Brown!15, fg=Sepia }
\setbeamercolor{item projected}{bg=Sepia, fg=White}
\setbeamercolor{number projected}{bg = Black}

%declara as imagens usadas no layout do slide
\pgfdeclareimage[height=0.8cm]{mflab}{figuras/logo_mflab_transparente.png}
\pgfdeclareimage[height=1.0cm]{logoufu}{figuras/logo_ufu.jpg}
\pgfdeclareimage[height=1.0cm]{petro}{figuras/petrobras_2.png}

%posiciona o logotipo do MFLab
\setlength{\TPHorizModule}{1mm}
\setlength{\TPVertModule}{1mm}
\newcommand{\placelogomflab} 
{ 
	\begin{textblock}{13}(150.0,0.0)
		\pgfuseimage{mflab} 
	\end{textblock} 
	
% 	\begin{textblock}{13}(128.0,1.0)
% 		\pgfuseimage{logoufu} 
% 	\end{textblock} 
	
	\begin{textblock}{13}(150.0,70.0)
		\pgfuseimage{petro} 
	\end{textblock} 
}
%posiciona o logotipo do MFLab
\setlength{\TPHorizModule}{1mm}
\setlength{\TPVertModule}{1mm}
\newcommand{\placelogo} 
{ 
	\begin{textblock}{13}(150.0,0.0)
		\pgfuseimage{mflab} 
	\end{textblock} 
	
% 	\begin{textblock}{13}(128.0,1.0)
% 		\pgfuseimage{logoufu} 
% 	\end{textblock} 
	
	\begin{textblock}{13}(0.0,80.0)
		\pgfuseimage{petro} 
	\end{textblock} 
}

% \setlength{\TPHorizModule}{1mm}
% \setlength{\TPVertModule}{1mm}
% \newcommand{\placelogomflab_titulo} 
% { 
% 	\begin{textblock}{13}(150.0,0.0)
% 		\pgfuseimage{mflab} 
% 	\end{textblock} 
% 	
% 	\begin{textblock}{13}(0.0,0.0)
% 		\pgfuseimage{lmest} 
% 	\end{textblock} 
% 	
% % 	\begin{textblock}{13}(128.0,1.0)
% % 		\pgfuseimage{logoufu} 
% % 	\end{textblock} 
% 	
% 	\begin{textblock}{13}(75.0,80.0)
% 		\pgfuseimage{petro} 
% 	\end{textblock} 
% }



%insere o logotipo da ufu em todos os slides
% \logo{\includegraphics[height=0.8cm]{figuras/layout_slide/petrobras.png}}

\title{Análise térmica em canal de Poiseuille turbulento: Um ajuste do número de Prandtl turbulento e valor de Cebeci para uma metodologia semi-exata}

\author{ Felipe J. O. Ribeiro \\ \and \\ Orientador: Prof. Dr. Aristeu da Silveira Neto}

%\date{\tiny{02 de dezembro de 2015}}
\date{\tiny{\today}}
% \newcolumntype{M}[1]{>{\raggedright\arraybackslash}b{#1}}
% \newcolumntype{N}{@{}m{0pt}@{}}	
% \newcolumntype{M}{>{\begin{minipage}[b]{3cm}\raggedright{}}c<{\end{minipage}\minrowheight}}
% \setlength\extrarowheight{5pt}
\newcolumntype{C}[1]{>{\centering\let\newline\\\arraybackslash\hspace{0pt}}m{#1}}


\begin{document}

	\begin{frame}\placelogomflab
		\frametitle 
		{ \vfill
			\centering
			{
			\small{Universidade Federal de Uberlândia}\\
%			\small{Programa de Pós-Graduação em Engenharia Mecânica}\\
			\small{Laboratório de Mecânica dos Fluidos}\\
			}
		}
		\maketitle
	\end{frame}

	\section<presentation>*{Sumário}
	
		\begin{frame}
			\frametitle{Sumário}\placelogomflab 
			{\scriptsize \tableofcontents}
		\end{frame}

		\AtBeginSection[]
		{
		 \begin{frame}<beamer>
		  \frametitle{Sumário}\placelogomflab 
		  {\scriptsize \tableofcontents[current,currentsection]}
		 \end{frame}
		}

		\AtBeginSubsection[]
		{
		 \begin{frame}<beamer>
		  \frametitle{Sumário}\placelogomflab 
		  {\scriptsize \tableofcontents[current,currentsubsection]}
		 \end{frame}
		}





	\section{Determinação teórica}
	
	
	
	
	
		\begin{frame}
			\frametitle{O canal do Poiseuille}
			$\bullet$ O problema foi definido como um fluxo de canal plano, com somente uma dimensão finita no eixo $y$. 
			$\bullet$ A condição de contorno foi determinada como duas placas infinitas em um regime de calor constante.
			$\bullet$ Um gradiente constante de pressão foi imposto somente no eixo $x$.\\
			$\bullet$ No eixo $z$ foi proposto uma auto similaridade no domínio da velocidade e da temperatura, resultando em uma análise bidimensional (Fig.\ref{figure.1}). \\
			$\bullet$ O fluido foi definido como Newtoniano e incompressível em regime turbulento.\\
			\begin{minipage}[h!]{0.2\textwidth}
				\begin{equation*}
				 \frac{\partial T }{\partial z} = 0
				\end{equation*}
				\begin{equation*}
				\frac{\partial u }{\partial z} = 0
				\end{equation*}
			\end{minipage}\hfill
			\begin{minipage}[h!]{0.75\textwidth}
			\begin{figure}[h!]
				\centering
				\includegraphics[angle=0, scale=0.40]{figure1}
				\caption{Definições geométricas e condições de contorno do sistema.}
				\label{figure.1}
			\end{figure}
			\end{minipage}
			\\
		\end{frame}
	
	
	
	
	
		\begin{frame}
			\frametitle{Equações do movimento}
		$\bullet$ Para modelar diferencialmente a velocidade do fluido, foram utilizadas a equação da continuidade e a equação de Navier-Stokes para a velocidade no eixo de interesse.
		\begin{equation}
		\frac{\partial u}{\partial t} + \frac{\partial u^2}{\partial x} + \frac{\partial uv}{\partial y} + \frac{\partial uw}{\partial z} = - \frac{1}{\rho} . \frac{\partial {p}}{\partial x} + \nu . \left( \frac{\partial^2 u}{\partial x^2} + \frac{\partial^2 u}{\partial y^2} + \frac{\partial^2 u}{\partial z^2}   \right)
		\end{equation}
		\begin{equation}
		\frac{\partial u}{\partial x} + \frac{\partial u}{\partial y} + \frac{\partial u}{\partial z} = 0
		\end{equation}
		Para o desenvolvimento da equação da energia em uma instancia advectiva foi necessário o desenvolvimento do perfil cinético do sistema.
		\end{frame}
	
	
		\begin{frame}
			\frametitle{Equação da energia}
		$\bullet$ Para a análise térmica, foram utilizadas a equação do balanço de energia e a equação da energia convectiva.
		\begin{equation}
		\frac{\partial T}{\partial t} + {\frac{\partial{}}{\partial{x}} (uT)} + 
		{\frac{\partial{}}{\partial{y}} (vT)} 
		=
		{\frac{\partial{}}{\partial{x}}} \left(\alpha {\frac{\partial{T}}{\partial{x}}} \right) +
		{\frac{\partial{}}{\partial{y}}} \left(\alpha {\frac{\partial{T}}{\partial{y}}} \right) 
		\end{equation}
		$ $
		$\bullet$ Esta segunda para propósitos de Adimensionalização.
		\begin{equation}\label{c_h_e}
		q_{conv.} = \dot{m} C_p \Delta T_m
		\end{equation}
		$ $
		Com estas construções matemáticas foi possível se iniciar o desenvolvimento diferencial.
		\end{frame}
	
	
	
	
	
	\section{Desenvolvimento diferencial}
		
		
		
		
		
		\begin{frame}
			\frametitle{Princípio dos valores médios}
			$\bullet$ Para se prosseguir com as simplificações do sistema advectivo foi necessário se utilizar deste conceito de valores médios. Tal consideração determina essa como uma metodologia RANS. (Reynolds Averaged Navier Stokes)
			\\
			\begin{minipage}[h!]{0.45\textwidth}
				\begin{equation*}
				\label{ola}
				\text{Simplificação}=
				\begin{cases}
				\overline{f}({x})=\frac{1}{t_f - t_i} \int_{t_i}^{t_f} f({x} , t) dt      & \quad  \\
				f({x} , t) = \overline{f}({x}) + f^\prime ({x} ,t)  & \quad   \\
				\overline{f^\prime ({x} ,t)} = 0  & \quad   \\
				\overline{\overline{f({x})}} = \overline{f({x})}  & \quad   \\
				\overline{f^\prime ({x} ,t)\overline{f({x})}} = 0  & \quad   \\
				\overline{f^\prime ({x} ,t)g^\prime ({x} ,t)} \neq 0  & \quad   \\
				\overline{  \overline{g({x})}. \overline{f({x})}  } = {\overline{g({x})}} . {\overline{f({x})}}  & \quad   \\
				\end{cases}
				\end{equation*}
			\end{minipage}\hfill
			\begin{minipage}[h!]{0.45\textwidth}
				\begin{figure}
					\centering
					\includegraphics[angle=0, scale=0.3]{medios}
					\caption{Representação gráfica do conceito.}
					\label{medios}
				\end{figure}
			\end{minipage}
	     	\\
		\end{frame}
	
	
	
	
	
		
				\begin{frame}
		\frametitle{Simplificando a velocidade para valores médios}
		\begin{equation}
		\frac{\partial \overline{u}}{\partial t} + \frac{\partial \overline{u^2}}{\partial x} + \frac{\partial \overline{uv}}{\partial y} + \frac{\partial \overline{uw}}{\partial z} = - \frac{1}{\rho} . \frac{\partial \overline{p}}{\partial x} + \nu . \left( \frac{\partial^2 \overline{u}}{\partial {x^2}} + \frac{\partial^2 \overline{u}}{\partial y^2} + \frac{\partial^2 \overline{u}}{\partial z^2}   \right)
		\end{equation}
		\\
		\begin{equation}
		\begin{split}
		\frac{\partial \overline{(\overline{u} + u^\prime)}}{\partial t} + \frac{\partial \overline{(\overline{u}^2 + 2 . \overline{u} . u^\prime + {u^\prime}^2)}}{\partial x} + \frac{\partial \overline{(\overline{u}.\overline{v} + u^\prime . \overline{v} + \overline{u} . v^\prime + u^\prime . v ^\prime )}}{\partial y} + \\
		\frac{\partial \overline{(\overline{u}.\overline{w} + u^\prime . \overline{w} + \overline{u} . w^\prime + u^\prime . w ^\prime )}}{\partial z} = - \frac{1}{\rho} . \frac{\partial \overline{(\overline{p} + p ^\prime)}}{\partial x} + \nu . \left( \frac{\partial^2 \overline{(\overline{u} + u^\prime)}}{\partial {x^2}} + \frac{\partial^2 \overline{(\overline{u} + u^\prime)}}{\partial y^2} + \frac{\partial^2 \overline{(\overline{u} + u^\prime)}}{\partial z^2}   \right)
		\end{split}
		\end{equation}
		\\
		\begin{equation}
		\begin{split}
		\frac{\partial \overline{u}}{\partial t} + \frac{\partial \overline{u^2}}{\partial x} + \frac{\partial \overline{u}.\overline{v}}{\partial y} + \frac{\partial \overline{u}.\overline{w}}{\partial z} =  - \frac{1}{\rho} . \frac{\partial \overline{{p}}}{\partial x} + \frac{\partial}{\partial x} \left( \nu.\frac{\partial \overline{u}}{\partial x} - \overline{{u^\prime}^2}\right) + \frac{\partial}{\partial y} \left( \nu.\frac{\partial \overline{u}}{\partial y} - \overline{{u^\prime . v^\prime}}\right) \\
		+ \frac{\partial}{\partial z} \left( \nu . \frac{\partial \overline{u}}{\partial z} - \overline{ u ^\prime . w ^\prime} \right)
		\end{split}
		\end{equation}
	\end{frame}
	
	
	
	
	
		\begin{frame}
		\frametitle{Simplificando a temperatura para valores médios}
		\begin{equation}
		\frac{\partial \overline{T}}{\partial t} + {\frac{\partial{}}{\partial{x}} \overline{(uT)}} + 
		{\frac{\partial{}}{\partial{y}} \overline{(vT)}} 
		=
		{\frac{\partial{}}{\partial{x}}} \left(\alpha {\frac{\partial{\overline{T}}}{\partial{x}}} \right) +
		{\frac{\partial{}}{\partial{y}}} \left(\alpha {\frac{\partial{\overline{T}}}{\partial{y}}} \right) 
		\end{equation}
		\begin{equation}\label{equation_mean}
		\frac{\partial \overline{T}}{\partial t} +{\frac{\partial{}}{\partial{x}} \overline{\left((\overline{u} + u^\prime) (\overline{T} + T^\prime) \right)}} + 
		{\frac{\partial{}}{\partial{y}} \overline{(\left(\overline{v} + v^\prime) (\overline{T} + T^\prime) \right)}} 
		=
		{\frac{\partial{}}{\partial{x}}} \left(\alpha {\frac{\partial{\overline{T}}}{\partial{x}}} \right) +
		{\frac{\partial{}}{\partial{y}}} \left(\alpha {\frac{\partial{\overline{T}}}{\partial{y}}} \right) 
		\end{equation}
		\begin{equation}
		\frac{\partial \overline{T}}{\partial t} +\frac{\partial{}}{\partial{x}} \left(\left(\overline{T^\prime u^\prime}\right) + \left(\overline{u}.\overline{T}\right)\right)     + 
		\frac{\partial{}}{\partial{y}} \left(\left(\overline{T^\prime v^\prime}\right) + \left(\overline{v}.\overline{T}\right)\right) 
		=
		{\frac{\partial{}}{\partial{x}}} \left(\alpha {\frac{\partial{\overline{T}}}{\partial{x}}} \right) +
		{\frac{\partial{}}{\partial{y}}} \left(\alpha {\frac{\partial{\overline{T}}}{\partial{y}}} \right) 
		\end{equation}
		\begin{equation}\label{equation_preparede}
		\frac{\partial \overline{T}}{\partial t} +\frac{\partial{}}{\partial{x}} \left(\overline{T^\prime u^\prime}\right) + \frac{\partial{}}{\partial{x}}\left(\overline{u}.\overline{T}\right)     + 
		\frac{\partial{}}{\partial{y}} \left(\overline{T^\prime v^\prime}\right) + \frac{\partial{}}{\partial{x}}\left(\overline{v}.\overline{T}\right) 
		=
		{\frac{\partial{}}{\partial{x}}} \left(\alpha {\frac{\partial{\overline{T}}}{\partial{x}}} \right) +
		{\frac{\partial{}}{\partial{y}}} \left(\alpha {\frac{\partial{\overline{T}}}{\partial{y}}} \right) 
		\end{equation}
		\end{frame}
		




		\begin{frame}
			\frametitle{Diferença de temperatura}
			$\bullet$ Apesar de já em valores médios, o domínio da temperatura ainda assim não se reduz a um problema unidimensional. Para isso, as condições de contorno devem ser controladas. No caso deste trabalho, estabeleceu-se um regime de calor constante, o que converteu-se em um gradiente linear de temperatura nas paredes no sentido do eixo $x$.  \\
			\begin{minipage}[h!]{0.32\textwidth}
				Tal gradiente linear se estende para todo o domínio, resultando em:
				\begin{equation}
				\frac{\partial \overline{T}}{\partial x} = ctt.
				\end{equation}
				Dessa forma, para se ter um sistema unidimensional representativo, se parametrizou a variável em função de variáveis de parede, ou seja:
				\begin{equation}
				\overline{T}(x,y) = \overline{T}^\ast(y) - \overline{T}_w(x)
				\end{equation}
			\end{minipage}\hfill
			\begin{minipage}[h!]{0.65\textwidth}
			\begin{figure}
				\centering
				\includegraphics[angle=0, scale=0.15]{imagemtermico}
				\caption{Representação gráfica do domínio térmico do sistema.}
				\label{temperatura}
			\end{figure}
			\end{minipage}
			
			
		\end{frame}
		
		
		
		
		
		
		\begin{frame}
		\frametitle{Desenvolvendo a diferença de temperatura na equação}
		\begin{equation}
		\begin{split}
		\frac{\partial{}}{\partial{x}} \left(\overline{(T^\ast + T_w)^\prime} \overline{ u^\prime}\right) + \frac{\partial{}}{\partial{x}}\left(\overline{(T^\ast + T_w)} \overline{u}\right)+ 
		\frac{\partial{}}{\partial{y}} \left(\overline{(T^\ast + T_w)^\prime} \overline{ v^\prime}\right) + \frac{\partial{}}{\partial{y}}\left(\overline{(T^\ast + T_w)} \overline{v}\right) = \\
		{\frac{\partial{}}{\partial{x}}} \left(\alpha {\frac{\partial{\overline{(T^\ast + T_w)}}}{\partial{x}}} \right) +
		{\frac{\partial{}}{\partial{y}}} \left(\alpha {\frac{\partial{\overline{(T^\ast + T_w)}}}{\partial{y}}} \right) 
		\end{split}
		\end{equation}
		\begin{center}\begin{equation}\begin{split}\label{oi}
		\frac{\partial{}}{\partial{x}} \left(\overline{T_w^{\prime} }. \overline{ u^{\prime}}\right) +\frac{\partial{}}{\partial{x}} \left(\overline{{T^{\ast}}^{\prime}}. \overline{ u^{\prime}}\right)
		+\frac{\partial{}}{\partial{x}}\left(\overline{u}. \overline{T^{\ast}}\right)+ 
		\frac{\partial{}}{\partial{x}}\left(\overline{u}. \overline{T_w}\right)+ 
		\\
		\frac{\partial{}}{\partial{y}} \left(\overline{{T^{\ast}}^{\prime}}. \overline{ v^{\prime}}\right)+
		\frac{\partial{}}{\partial{y}} \left(\overline{T_w^\prime}. \overline{ v^\prime}\right) + \frac{\partial{}}{\partial{y}}\left(\overline{v}. \overline{T^\ast}\right) +
		\frac{\partial{}}{\partial{y}}\left(\overline{v}. \overline{T_w}\right) 
		= 
		\\
		{\frac{\partial{}}{\partial{x}}} \left(\alpha {\frac{\partial{\overline{(T^\ast + T_w)}}}{\partial{x}}} \right) +
		{\frac{\partial{}}{\partial{y}}} \left(\alpha {\frac{\partial{\overline{(T^\ast + T_w)}}}{\partial{y}}} \right) 
		\end{split}\end{equation}\end{center}
		\end{frame}
		
		
		
		
		
		\begin{frame}
			\frametitle{Simplificações chave na velocidade}
			$\bullet$ Para a velocidade, devem ser feitas as seguintes considerações:
			\begin{equation}
			\frac{\partial \overline{u}}{\partial x} = 0
			\end{equation}
			\begin{equation}
			\overline{v} = \overline{w} = 0
			\end{equation}
			\begin{equation}
			\frac{\partial \overline{u}}{\partial t}
			\end{equation}
			Assim, na equação da velocidade, teremos as seguintes simplificações:
			\begin{equation}
			\begin{split}
			\color{red}\frac{\partial \overline{u}}{\partial t} \color{black}+\color{red} \frac{\partial \overline{u^2}}{\partial x} \color{black}+\color{red} \frac{\partial \overline{u}.\color{blue}\overline{v}\color{red}}{\partial y}\color{black} + \color{red} \frac{\partial \overline{u}.\color{blue}\overline{w}\color{red}}{\partial z} \color{black} =  - \frac{1}{\rho} . \frac{\partial \overline{{p}}}{\partial x} + \frac{\partial}{\partial x} \left( \nu. \color{red}\frac{\partial \overline{u}}{\partial x} \color{black} - \overline{{u^\prime}^2}\right) + \frac{\partial}{\partial y} \left( \nu.\frac{\partial \overline{u}}{\partial y} - \overline{{u^\prime . v^\prime}}\right) \\
			+ \color{red}\frac{\partial}{\partial z} \left( \nu . \frac{\partial \overline{u}}{\partial z} - \overline{ u ^\prime . w ^\prime} \right) \color{black}
			\end{split}
			\end{equation}
			
				
			
		\end{frame}
		
		
		
		
		
		\begin{frame}
		\frametitle{Simplificações chave na temperatura}
		\begin{equation*}
		\color{red}{\frac{\partial \overline{T}}{\partial t}} \color{black} +\frac{\partial{}}{\partial{x}} \left(\overline{T^\prime u^\prime}\right) + \frac{\partial{}}{\partial{x}}\left(\overline{u}.\overline{T}\right)     + 
		\frac{\partial{}}{\partial{y}} \left(\overline{T^\prime v^\prime}\right) + \frac{\partial{}}{\partial{x}}\left(\color{red}\overline{v} \color{black}.\overline{T}\right) 
		=
		{\frac{\partial{}}{\partial{x}}} \left(\alpha {\frac{\partial{\overline{T}}}{\partial{x}}} \right) +
		{\frac{\partial{}}{\partial{y}}} \left(\alpha {\frac{\partial{\overline{T}}}{\partial{y}}} \right) 
		\end{equation*}
		\end{frame}
		
		
		
		
		
		\begin{frame}
			\frametitle{Hipótese de Boussinesq}
		\end{frame}
	
	
	
	
		
		\begin{frame}
			\frametitle{Comprimento de mistura de Prandtl}
		\end{frame}
	
	
	
	
	
		\begin{frame}
			\frametitle{Balanço de energia}
		\end{frame}
	
	
	
	
	
		\begin{frame}
			\frametitle{Adimensionalização}
		\end{frame}
	


		
		
	\section{Resultados}
		
		


		\begin{frame}
			\frametitle{Perspectivas}
			$\bullet$ Extrapolação temporal para chute inicial do acoplamento forte;\\
			$\bullet$ Spool no AMR;\\
			$\bullet$ Validação do acoplamento forte;\\
			$\bullet$ Outro método de fronteira imersa (Ghost Fluid);\\
			$\bullet$ Estimativa do Jacobiano via mínimos quadrados;\\
			$\bullet$ Fluido-estrutura monolítico:\\
			$\longrightarrow$ Fluido monolítico;\\
			$\longrightarrow$ Ghost Fluid monolítico;\\
			$\longrightarrow$ Modelo de turbulência monolítico;\\
			$\longrightarrow$ Estrutura monolítico.\\
		\end{frame}	
	
	
	
	
	
	
	
	\section{Agradecimentos}
		
		
		
		
		
			\begin{frame}
				\placelogomflab 
				\frametitle{Agradecimentos}
				\begin{figure}
					\begin{center}
						\begin{tabular}{c c}
							{\includegraphics[trim=0.0cm 0.0cm 0.0cm 0.0cm,clip=true,height=0.2\textheight]{figuras/petrobras.png}}&{\includegraphics[trim=0.0cm 0.0cm 0.0cm 0.0cm,clip=true,height=0.2\textheight]{figuras/logo_mflab.png}}\\
							{\includegraphics[trim=0.0cm 0.0cm 0.0cm 0.0cm,clip=true,height=0.2\textheight]{figuras/cnpq.png}}&{\includegraphics[trim=0.0cm 0.0cm 0.0cm 0.0cm,clip=true,height=0.2\textheight]{figuras/CAPES.png}}\\
							{\includegraphics[trim=0.0cm 0.0cm 0.0cm 0.0cm,clip=true,height=0.2\textheight]{figuras/FAPEMIG.jpg}}&{\includegraphics[trim=0.0cm 0.0cm 0.0cm 0.0cm,clip=true,height=0.2\textheight]{figuras/UFU_black.jpg}}\\
						\end{tabular}
					\end{center}
				\end{figure}
			\end{frame}
			
			
			
			
			
			\begin{frame}
				\placelogomflab 
				\frametitle{Agradecimentos}
				\fontsize{44pt}{7.2}\selectfont
				\begin{center}
					Obrigado.
				\end{center}
			\end{frame}
		
		
		
		
\end{document}




		



%%%%%%%%%%%%%%%%%%%%%%%%%%%%%%%%%%%%%%% Exemplo de formatação de imagens		
%		\begin{frame}
%			\frametitle{Adição de fronteiras extras}
%			\begin{tabular}{c c}
%				
%				{\includegraphics[trim=0.0cm 0.0cm 0.0cm 0.0cm,clip=true,loop,height=0.5\textheight]{figuras/filtration_depois.png}}&{\includegraphics[trim=0.0cm 0.0cm 0.0cm 0.0cm,clip=true,loop,height=0.4\textheight]{figuras/filtration_depois_zoom.png}}\\
%				
%			\end{tabular}
%			
%		\end{frame}




%%%%%%%%%%%%%%%%%%%%%%%%%%%%%%%%%%%%%% Exemplo de formatação de imagens		
%		\begin{frame}
%			\frametitle{Agora}
%			\centering
%			\begin{tabular}{c}
%				
%				{\includegraphics[trim=0.00cm 2.0cm 0.0cm 2.0cm,clip=true,loop,width=0.9\textwidth]{figuras/t_x_51f.png}}\\{\includegraphics[trim=0.01cm 0.0cm 0.01cm 0.0cm,clip=true,loop,width=0.9\textwidth]{figuras/t_x_51999.png}}\\{\includegraphics[trim=0.01cm 0.0cm 0.01cm 0.0cm,clip=true,loop,width=0.9\textwidth]{figuras/t_x_51999g.png}}\\{\includegraphics[trim=0.01cm 0.0cm 0.01cm 0.0cm,clip=true,loop,width=0.9\textwidth]{figuras/t_x_51999y.png}}\\{\includegraphics[trim=0.01cm 0.0cm 0.01cm 0.0cm,clip=true,loop,width=0.9\textwidth]{figuras/t_x_51999b.png}}
%				
%			\end{tabular}
%			
%		\end{frame}





%%%%%%%%%%%%%%%%%%%%%%%%%%%%%%%%%%%%%  Formatação de equações:		
%		\begin{frame}
%			\frametitle{Newton-Raphson}
%			
%			\flushleft
%			Método de interface com jacobiano composto:
%			
%			\centering
%			\begin{equation}\label{forte_eqNewton}
%			K(D+\Delta D) \approx K(D)+\Delta D \, J(D)
%			\end{equation}
%			\begin{equation}\label{forte_eqNewton2}
%			K(D) =  Estrutura(Fluido(D))-D =  0
%			\end{equation}
%			\begin{equation}\label{forte_eqNewton3}
%			J(D) =  Estrutura'(Fluido(D)) \, Fluido'(D)-I
%			\end{equation}
%			\begin{equation}\label{forte_eqNewton4}
%			Fluido(D): \mathbb{R}^{n} \to \mathbb{R}^{m}
%			\end{equation}
%			
%			\flushleft
%			$Fluido'(D)$ é de tamanho $m x n$
%			
%			\centering
%			
%			\begin{equation}\label{forte_eqNewton5}
%			Estrutura(F): \mathbb{R}^{m} \to \mathbb{R}^{n}
%			\end{equation}
%			
%			\flushleft
%			$Estrutura'(F)$ é de tamanho $n x m$\\
%			$Estrutura'(Fluido(D)) \, Fluido'(D)$ e $I$ é de tamanho $n x n$
%		\end{frame}




%%%%%%%%%%%%%%%%%%%%%%%%%%%%%%%%%%%%%%%%%%% Vários exemplos de formatação textual:		


%		\begin{frame}
%			\frametitle{Conveniência do método de Multi Direct Forcing}
%			
%			\flushleft
%			\textbf{Fraco:}\\
%			$\bullet$ Predição da velocidade.\\
%			$\bullet$ MDF. (Imposição da condição de dirichlet na interface e cálculo da força)\\
%			$\bullet$ Estrutura.\\
%			$\bullet$ Poisson.\\
%			$\bullet$ Correção de velocidade e pressão.\\ \\
%
%			\textbf{Forte:}\\
%			$\bullet$ Predição da velocidade.\\
%			while \\
%			\quad	$\longrightarrow$ MDF.\\
%			\quad	$\longrightarrow$ Estrutura.\\
%			end\\
%			$\bullet$ Poisson.\\
%			$\bullet$ Correção de velocidade e pressão.\\
%
%		\end{frame}

		

%		
%%%%%%%%%%%%%%%%%%%%%%%%%%%%%%%%%  Modelo duas fotos lado a lado:


%		\begin{frame}
%		\frametitle{Limite do fraco}
%			ct=121
%			mi=200
%			\begin{tabular}{c c}
%			{\includegraphics[width=0.45\linewidth]{../../simulacoes_Estudo_dirigido2/fraco_mi_200_0_15_ct141/figuras/estrutura/vel_151}}&
%		   {\includegraphics[width=0.45\linewidth]{../../simulacoes_Estudo_dirigido2/fraco_mi_200_0_15_ct141/figuras/estrutura/vel_251}}\\
%		   {(a) Velocidade em linha centro da estrutura} & {(b) Velocidade transversal centro da estrutura}
%		\end{tabular}
%		\end{frame}



%%%%%%%%%%%%%%%%%%%%%%%%%%%%%%%%%%  Modelo tabela :

%		\begin{frame}
%			\frametitle{Comparação número de iterações}
%			\begin{tabular}{c c c c}
%				\hline
%				Método & Mínimo     &    Máximo &  Média\\ \hline
%				FPI MDF variável & 8     &    101 &  8.9764764764764760\\
%				FPI MDF fixo & 8     &     11 &  8.9099099099099099\\
%				QN Primeiro método de Broyden MDF variável & 18    &     101 &  18.281281281281281 \\ \hline
%			\end{tabular}
%		\end{frame}	




