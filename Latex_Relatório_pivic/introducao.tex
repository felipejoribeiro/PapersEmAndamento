\chapter{Introdução}

\noindent 

 
 	O estudo do comportamento térmico de escoamentos é de suma importância ao desenvolvimento científico atual. A medida que o maquinário industrial é aperfeiçoado, também cresce o consumo energético no mundo. Grande parte deste custo surge das transformações energéticas dentro do sistema, em sua maioria, resultando em manifestações térmicas. Assim surge uma grande necessidade de mecanismos de gerenciamento térmico. Como a difusão é um processo muito lento, o meio mais utilizado para se resfriar maquinários industriais é o advectivo, onde uma interface fluido-estrutura carrega energia térmica para fora do sistema. Estudar estes fenômenos é essencial para que hajam cada vez máquinas mais eficientes, assim assegurando um desenvolvimento sustentável da sociedade. 
 	
	Simular fluxos turbulentos é uma tarefa complexa. Envolve comportamentos não lineares e, na maioria dos casos, não pode ser resolvido algebricamente. Tal complexidade é bem descrita em John C. Sommerer (1997), onde se dizia que o aspecto caótico do problema o torna um assunto muito difícil de se estudar. Com isso em mente, os métodos de simplificação são uma preocupação constante para este campo, pois tal não linearidade gera a necessidade de grandes esforços computacionais.
	Neste sentido, no presente artigo, desenvolve-se um RANS (Reynolds Averaged Navier-Stokes) unidimensional que soluciona o campo de temperatura em um canal turbulento de Poiseuille (Poiseuille, 1846). Tal abordagem de valores médios procura acurácia, mas sem se deixar cair nos esforços computacionais exacerbados de um DNS (Direct Numerical Simulation).
	
	Para resolver o campo de temperatura, a equação de energia térmica foi acoplada com o fenômeno da convecção. Por isso, o campo de velocidade tinha que ser resolvido, o que traz a equação de Navier-Stokes para a abordagem matemática. Foi necessária a modelagem dos termos não-lineares nas expressões para simplificar essas equações e fechar o modelo.
	Estudos experimentais como Nikuradse (1966), Van Driest (1956) foram utilizados, assim como hipóteses conceituais, como a de Boussinesq (1877), e também métodos puramente computacionais como algoritmos genéticos para o ajusto do número de Prandtl turbulento de forma multi objetiva. 
	Detalhes são apresentados e discutidos.
	
	
\newpage
	

 \section{Metodologia}
 % --------------------------------------------------------------------------

\noindent

	A metodologia proposta para o presente trabalho de iniciação científica se baseia
na definição de problemas, estudando primeiramente os casos particulares, e uma vez
compreendida a natureza do conteúdo, seguindo para problemas mais gerais e complexos.

	Inicialmente verifica-se na literatura as dinâmicas envolvidas nos mecanismos propostos para este estudo. A partir das relações obtidas nessa revisão, elabora-se uma discretização a partir de métodos numéricos e realiza-se a implementação dos mesmos em um código computacional. As rotinas são confeccionadas na linguagem de programação Fortran, para familiarização do aluno com a linguagem base dos códigos do Laboratório de Mecânica dos Fluidos Computacional (MFLab) da Faculdade de Engenharia Mecânica (FEMEC) da Universidade Federal de Uberlândia.
 

\subsection{Difusão}

\noindent

	A difusão é um mecanismo de transporte que possui diferentes definições nos campos da química, biologia e física. Especificamente para o caso de transferência de calor, a difusão é bem definida matematicamente por uma equação diferencial parcial (EDP), que envolve uma derivada parcial de primeira ordem no domínio temporal e uma derivada parcial de segunda ordem no domínio espacial. Os termos são relacionados, através de uma constante (difusividade), à um termo fonte, resultante da diferença desses elementos, como indicado na Eq. (\ref{Difusaoone}).
	
\begin{align}
 \label{Difusaoone}
 f(x,y,z,t) = \dfrac{\partial \phi}{\partial t} - \alpha \nabla^2 \phi
\end{align}

	Assim, para o caso unidimensional, obtém-se a relação dada pela Eq.(\ref{Difuni}).
	
\begin{align}
 \label{Difuni}
 f(x,t) = \dfrac{\partial \phi}{\partial t} - \alpha \left(\dfrac{\partial^2 \phi}{\partial x^2}\right)
\end{align}

	E para o caso bidimensional, obtém-se a relação dada pela Eq.(\ref{Difbi}).
	
\begin{align}
 \label{Difbi}
 f(x,y,t) = \dfrac{\partial \phi}{\partial t} - \alpha \left[ \left(\dfrac{\partial^2 \phi}{\partial x^2}\right) - \left(\dfrac{\partial^2 \phi}{\partial y^2}\right)\right]
\end{align}
	
	A difusividade ($\alpha$) é traduzida fisicamente como a rapidez com que a energia térmica é transportada espacialmente para dado material. O termo fonte ($f$), por sua vez, se traduz na resultante da diferença das parciais, que deve ser nulo para a solução da difusão. 

\subsection{Advecção}

\noindent

	A advecção é um mecanismo de transporte que também é presente em vários campos de estudo, dado pela transferência de calor juntamente com a transferência de espécie. Esse fenômeno é modelado matemáticamente pela Eq. (\ref{Adveccao}).
	
\begin{align}
\label{Adveccao}
f(x,y,z,t) = \dfrac{\partial \phi}{\partial t} + c \nabla \phi
\end{align}

	Para o caso unidimensional, de forma análoga ao caso da difusão, tem-se que a relação fica como indicada pela Eq. (\ref{advuni}).
	
\begin{align}
\label{advuni}
f(x,t) = \dfrac{\partial \phi}{\partial t} + c \dfrac{\partial \phi}{\partial x}
\end{align}

	Para o caso bidimensional, obtém-se a relação dada pela Eq. (\ref{advbi}).
	
\begin{align}
\label{advbi}
f(x,t) = \dfrac{\partial \phi}{\partial t} + cx \dfrac{\partial \phi}{\partial x} + cy \dfrac{\partial \phi}{\partial y}
\end{align}
	
	A velocidade ($c$) é traduzida fisicamente como a velocidade com que a variação térmica ocorre com a movimentação de massa na direção avaliada, sendo que essa velocidade pode ser positiva ou negativa. Já o termo fonte ($f$), se traduz na resultante da soma das parciais, que deve ser nulo para a solução da difusão.

\subsection{Difusão e Advecção}
\noindent

	O efeito combinado dos dois mecanismos é obtido de forma intuitiva através da soma dos termos difusivos e advectivos. Pode-se então descrever o fenômeno através da Eq.(\ref{difadv}).
	
\begin{align}
\label{difadv}
f(x,y,z,t) = \dfrac{\partial \phi}{\partial t} - \alpha \nabla^2 \phi + c \nabla \phi
\end{align}

	Novamente, para o caso unidimensional, simplesmente exclui-se dois termos espaciais, resultando na Eq.(\ref{difaduni}).

\begin{align}
\label{difaduni}
f(x,t) = \dfrac{\partial \phi}{\partial t} - \alpha \left(\dfrac{\partial^2 \phi}{\partial x^2}\right) + c \dfrac{\partial \phi}{\partial x}
\end{align}	

	Finalmente, para o caso bidimensional, exclui-se apenas um termo espacial, resultando na Eq.(\ref{difadbi}).

\begin{align}
\label{difadbi}
f(x,t) = \dfrac{\partial \phi}{\partial t} - \alpha \left[ \left(\dfrac{\partial^2 \phi}{\partial x^2}\right) + \left(\dfrac{\partial^2 \phi}{\partial y^2}\right)\right] + cx \dfrac{\partial \phi}{\partial x} + cy \dfrac{\partial \phi}{\partial y}
\end{align}	
	
\subsection{Malha}
\noindent

	Para este trabalho a malha é não adaptativa, devido ao baixo custo operacional das rotinas. Então, faz-se uso da condição de Courant-Friedrichs-Lewy (CFL) para a confecção da mesma.
	
	A CFL é uma condição necessária para a solução de certas equações diferenciais parciais pelo método de diferenças finitas. Tal condição é obtida de uma análise do diferencial de tempo explícito em relação ao diferencial espacial. Como conclusão, percebe-se que proporções maiores que aquelas ditadas pelo CFL correspondente resultam em sistemas instáveis e não convergentes.
	
	Para o caso da difusão, o passo temporal se relaciona ao passo espacial como indicado pela Eq.(\ref{Cflone}) para o caso unidimensional, e pela Eq.(\ref{Cfltwo}) para o caso bidimensional.
	
\begin{align}
\label{Cflone}
\Delta t = CFL \dfrac{(\Delta x)^2}{\alpha}
\end{align}

\begin{align}
\label{Cfltwo}
\Delta t = min \left(CFL \dfrac{(\Delta x)^2}{\alpha} , CFL \dfrac{(\Delta y)^2}{\alpha}\right)
\end{align}

	Para o caso da advecção, o passo temporal se relaciona ao passo espacial como indicado pela Eq.(\ref{Cflthree}) para o caso unidimensional, e pela Eq.(\ref{Cflfour}) para o caso bidimensional.
	
\begin{align}
\label{Cflthree}
\Delta t = CFL \dfrac{\Delta x}{\mid c \mid}
\end{align}

\begin{align}
\label{Cflfour}
\Delta t = min \left(CFL \dfrac{\Delta x}{\mid cx \mid}, CFL \dfrac{\Delta y}{\mid cy \mid}\right)
\end{align}

	Ainda, para os efeitos conjulgados, os passos temporal e espacial se relacionam como indicado pela Eq.(\ref{Cflfive}) para o caso unidimensional e como indicado pela Eq.(\ref{Cflsix}) para o caso bidimensional.
	
\begin{align}
\label{Cflfive}
\Delta t = min \left(CFL \dfrac{(\Delta x)^2}{\alpha}, CFL \dfrac{\Delta x}{\mid c \mid}\right)
\end{align}

\begin{align}
\label{Cflsix}
\Delta t = min \left(CFL \dfrac{(\Delta x)^2}{\alpha} , CFL \dfrac{(\Delta y)^2}{\alpha}, CFL \dfrac{\Delta x}{\mid cx \mid}, CFL \dfrac{\Delta y}{\mid cy \mid}\right)
\end{align}

	A malha para este trabalho é composta de células de dimensões $\Delta x$ por $\Delta y$, nucleadas em seu centro. A coordenada utilizada para as operações são referentes aos núcleos. 
	
Para um dado domínio em uma direção qualquer, sabe-se que as condições de contorno devem ser aplicadas nas arestas da célula. Para que isso ocorra, pode-se trabalhar na suposição de uma célula fantasma, que aplicada no método numérico gera a condição de contorno na parede, ou simplesmente usar a metade de uma célula para o contorno do domínio. A modelagem matemática numérica para os problemas aqui propostos é baseada na segunda condição. Assim, as células possuem dimensões fracionadas na fronteira.
	Uma representação de uma malha não adaptativa de duas dimensões é ilustrada a seguir na Figura(1.1).
\newline
\newline	

\begin{figure}[ht!]
	\label{malhim}
	\centering
	\includegraphics[width=60mm]{Imagens/malha.png}
	\caption{Exemplo de uma malha não adaptativa}
\end{figure}