\chapter[Conclusão]{Conclusão}
Neste trabalho foi desenvolvido um manual para guiar os alunos dos cursos de graduação no aprendizado de mecânica dos fluidos. A prática e o ensino ativo é estimulado com casos canônicos que introduzem o aluno aos temas centrais da matéria. O software utilizado foi o MFGui, que é a interface gráfica do MFSim, programa de computação fluido-dinâmica desenvolvido nos últimos 10 anos na Universidade Federal de Uberlândia. Espera-se que, com este guia, o ensino de mecânica dos fluidos torne-se mais acessível e instrutivo.

\section{Principais Contribuições}

A linguagem tem como objetivo ser convidativa e simples, com capítulos em formato de receita, concebido em passos claros e objetivos, com explicações teóricas para justificar cada instrução.