\chapter[Conclusão]{Conclusão}

No presente trabalho, os autores desenvolveram de forma bem sucedida um métod semi-analítico para o cálculo do campo de temperatura sobre um escoamento turbulento em canal de Poiseuille, validando-se os resultados com os de DNS. Durante o estudo, observou-se a influência que os parâmetros da constante de Cebeci e o número de Prandtl turbulento tinham sobre os resultados, e constatou-se que o valor canônico do Prandtl turbulento de $0.7$, que fora usado nas simulações não correspondia aos observados nas simulações de DNS, que variavam não só com o número de Reynolds turbulento, mas também com a distância da parede.

A partir dessas observações, novos modelos foram criados para descrever essa grandeza de forma que tornou-se os resultados do presente estudo mais representativos. Tais modelos ajustados foram criados a partir de métodos de evolução diferencial, que otimizaram o sistema para a menor norma L2 quando comparado com a solução numérica em DNS.

Os resultados obtidos com o modelo ajustado foram satisfatórios, e mostraram-se mais próximos dos resultados em DNS que a simulação desenvolvida usando-se o número de Prandtl turbulento advindo da própria solução DNS, o que mostra que o método de otimização ajustou o sistema de forma a corrigir distorções advindas não só do Prandtl turbulento, mas também de outros ajustes feitos durante o desenvolvimento do método semi-analítico.
