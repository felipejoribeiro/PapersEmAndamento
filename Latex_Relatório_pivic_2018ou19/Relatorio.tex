%%%%%%%%%%%%%%%%%%%%%%%%%%%%%%%%%%%%%%%%%%%%%%%%%%%%%%%%%%%%%%%%%%%%%%%%%%%%%%%%%%%%%%%%%%%%%%%%%%%%%%
   \documentclass[a4paper,12pt]{report}                   % Tipo de documento
   %\documentclass[10pt,twoside,a4paper]{article}         % Para duas ou mais colunas, pesquise a respeito
   \usepackage[utf8]{inputenc}                            % Biblioteca para ascentos 
   \usepackage[brazil]{babel}                             % Biblioteta para português
   \usepackage{amssymb}                                   % Biblioteta para simbolos
   \usepackage{graphicx}                                  % Biblioteta para figuras 
   \usepackage{amsmath}                                   % Biblioteta para simbolos matemáticos
   \usepackage{esint}                                     % Biblioteta para ...
   \usepackage{enumitem}                                  % Biblioteta para numerar itens
   \usepackage{amsfonts}                                  % Biblioteta para mais simbolos
   \usepackage{amscd}                                     % Biblioteta para ...
   \usepackage{amstext}                                   % Biblioteta para ...
   \usepackage{mathrsfs}                                  % Biblioteta para ambiente matematico
   \usepackage{amsthm}                                    % Biblioteta para ...
   \usepackage{hyperref}                                  % Biblioteta para alguma coisa importante
   \usepackage{wrapfig}                                   % Biblioteta para colocar duas figuras
   \usepackage{textcomp}                                  % Biblioteta para ...
   \usepackage{lipsum}                                    % É importante também
   \usepackage{longtable}                                 % Biblioteta para tabelas
   \usepackage{multicol}                                  % Biblioteta para tabelas
   \usepackage{geometry}  
   \usepackage{setspace} 
   \usepackage{subcaption}
   \usepackage{lettrine}
   \usepackage{fouriernc}                                      % New Century Schoolbook font
   \usepackage{xcolor}                                         % Controle de cores
   
   
\hypersetup{
	colorlinks=true,
	linkcolor=blue,
	filecolor=magenta,      
	urlcolor=cyan,
}
   
   
   %\usepackage[latin1]{inputenc}                         % para acentuação em Português OU
   %\usepackage[utf8]{inputenc}                           % para acentuação em Português com o uso do Unicode,  mude a codificação desse template para utf-8
   %\usepackage[english]{babel}                           % para texto em Inglês
   %\usepackage{graphics}                                 % Para colcar figuras
   %\usepackage{subfigure}                                % Para colocar texto e figuras     
   %\usepackage{epsfig}                                   % Para colocar figuras em eps
   %\usepackage[centertags]{amsmath}                      % Para mudar a fonte do ambiente matemático
   %\usepackage{indentfirst}                              % Para Parágrafos
   %\usepackage{newlfont}                                 % Não sei 
   %\usepackage{cite}                                     % Para fazer citações 'mais bonitas'
   %\usepackage[usenames,dvipsnames]{color}               % Alguma coisa com cor
   %\usepackage[T1]{fontenc}                              % Alguma fonte
   %\usepackage[a4paper]{geometry}                   	  % Define o tamanho da folha
   %\usepackage{subcaption}                               % Figuras lado a lado
   %\usepackage{booktabs}                                 % Tabelas
   %\usepackage{url}                                      % Web links
   %\usepackage{lipsum}                                   % Texto automático
   %\usepackage{wasysym}                                  % Biblioteca para simbolos
   %\usepackage{MnSymbol}                                 % Sim, mais simbolos
   %\usepackage{helvet}                                   % Fonte
   %\usepackage{lineno}                                   % Pacote para colocar números das linhas
   %\usepackage{mathptmx}                                 % Fonte padrão em fórmulas - Times!!!!
   %\usepackage{fancyhdr}                                 % Para editar cabeçalhos
   %\usepackage{latexsym}                                 % Simbolos ...
%%%%%%%%%%%%%%%%%%%%%%%%%%%%%%%%%%%%%%%%%%%%%%%%%%%%%%%%%%%%%%%%%%%%%%%%%%%%%%%%%%%%%%%%%%%%%%%%%%%%%%
   
%%%%%%%%%%%%%%%%%%%%%%%%%%%%%%%%%%%%%%%%%%%%%%%%%%%%%%%%%%%%%%%%%%%%%%%%%%%%%%%%%%%%%%%%%%%%%%%%%%%%%%
   % Define o scopo do texto (CAPA)
   %\title{Relatório final de iniciação científica}  
   %\author{Ítalo Augusto Magalhães de Ávila}
%%%%%%%%%%%%%%%%%%%%%%%%%%%%%%%%%%%%%%%%%%%%%%%%%%%%%%%%%%%%%%%%%%%%%%%%%%%%%%%%%%%%%%%%%%%%%%%%%%%%%%%
   
%%%%%%%%%%%%%%%%%%%%%%%%%%%%%%%%%%%%%%%%%%%%%%%%%%%%%%%%%%%%%%%%%%%%%%%%%%%%%%%%%%%%%%%%%%%%%%%%%%%%%%%
   % Seta o scopo do texto
   
    %\geometry{tmargin=1.2cm,bmargin=1.2cm,lmargin=1.2cm,rmargin=1.2cm}     % Define as margens
   
   \setlength{\textwidth}{15cm}                                 % Lateral
   \setlength{\textheight}{22cm}                                % Altura
   
   %\pagestyle{fancy}                                           % Coloca cabeçalho e rodapé
   %\fancyhead{}                                                % Tira informações do cabeçalho, como seção, etc
   %\fancyfoot{}                                                % Tira número das páginas do rodapé
   %\renewcommand{\headrulewidth}{0pt}                          % Retira linha automática de cima
   
%%%%%%%%%%%%%%%%%%%%%%%%%%%%%%%%%%%%%%%%%%%%%%%%%%%%%%%%%%%%%%%%%%%%%%%%%%%%%%%%%%%%%%%%%%%%%%%%%%%%%%%
   
%%%%%%%%%%%%%%%%%%%%%%%%%%%%%%%%%%%%%%%%%%%%%%%%%%%%%%%%%%%%%%%%%%%%%%%%%%%%%%%%%%%%%%%%%%%%%%%%%%%%%%%
   % Definição de comandos
   \newcommand{\sen}{{\rm \: sen\: }}                      % Seno   
   \newcommand\x{\times}                                   % Multiplicação Vetorial
   \newcommand\bigzero{\makebox(0,0){\text{\huge0}}}       % Um Zero bem grande
   \newcommand*{\bord}{\multicolumn{1}{c|}{}}              % Linha Vertical
   
   \newcommand{\pr}{\hspace*{.6cm}}                        % Parágrafo, usar apenas no início da seção ou subseção
   
   \renewcommand{\qedsymbol}{$\blacksquare$}               % C.Q.D.
   \newcommand{\conjC}{\mathbb{C}}                         % Conjuntos
   \newcommand{\conjR}{\mathbb{R}}	                       % Conjuntos
   \newcommand{\conjN}{\mathbb{N}}                         % Conjuntos
   \newcommand{\conjT}{\mathbb{T}}                         % Conjuntos
%%%%%%%%%%%%%%%%%%%%%%%%%%%%%%%%%%%%%%%%%%%%%%%%%%%%%%%%%%%%%%%%%%%%%%%%%%%%%%%%%%%%%%%%%%%%%%%%%%%%%%%
   
%%%%%%%%%%%%%%%%%%%%%%%%%%%%%%%%%%%%%%%%%%%%%%%%%%%%%%%%%%%%%%%%%%%%%%%%%%%%%%%%%%%%%%%%%%%%%%%%%%%%%%%
   % Definição de Novas secções e Teoremas
   \newtheorem{teo}{\normalcolor Teorema}                      % Para colocar um Teorema
   \newtheorem{col}[teo]{\normalcolor Corol\'ario}             % Para colocar um Colorário
   \newtheorem{prop}[teo]{\normalcolor Proposi\c c\~ao}        % Para colocar uma Preposição
   \newtheorem{definicao}{\normalcolor Defini\c{c}\~{a}o}      % Para Definir algo
   \newtheorem{exemplo}{\normalcolor Exemplo}                  % Para dar Exemplos
   \newtheorem{solucao}{\normalcolor solu\c{c}\~{a}o}          % Para Solução de Exemplos
   \newtheorem{lema}[teo]{\normalcolor Lema}                   % Para colocar um Lema
   \newtheorem{Demc}{Demonstração}[teo]                        % Para Demonstrar um Teoremas
   \newtheorem{Dem}{Demonstração}[teo]                         % Uai, é a mesma coisa do de cima
   \newtheorem{nota}[teo]{Nota}                                % Para fazer uma pequena observação
   \newtheorem{obs}[teo]{Observação}                           % A mesma coisa
   
   \usepackage[pages=some,scale=1,angle=0,opacity=1]{background} % imagem de Background
   \newcommand{\plogo}{\fbox{  $EPTA$   }}                       % Logo da equipe (template)
   \newcommand\BackImage[2][scale=1]{                            % cria a função background
   	\BgThispage
   	\backgroundsetup{
   		contents={\includegraphics[#1]{#2}}
   	}
   }
   
%%%%%%%%%%%%%%%%%%%%%%%%%%%%%%%%%%%%%%%%%%%%%%%%%%%%%%%%%%%%%%%%%%%%%%%%%%%%%%%%%%%%%%%%%%%%%%%%%%%%%%%
   
  \begin{document}

     %\maketitle    	 
  	  
  	  
    
  	 \begin{titlepage}
	\begin{center}
	
	%\begin{figure}[!ht]
	%\centering
	%\includegraphics[width=2cm]{c:/ufba.jpg}
	%\end{figure}

		\Huge{Universidade Federal de Uberlândia}\\
		\large{Faculdade de Engenharia Mecânica}\\ 
		\large{Laboratório de Mecânica dos Fluidos Computacional}\\ 
		\vspace{15pt}
        \vspace{95pt}
        \textbf{\LARGE{Introdução ao estudo diferencial analítico e à metodologia computacional em
mecânica dos fluidos}}\\
		%\title{{\large{Título}}}
		\vspace{7,5cm}
	\end{center}
	
	\begin{flushleft}
		\begin{tabbing}
			Aluno: Ítalo Augusto Magalhães de Ávila \\
			Prof. orientador: Aristeu da Silveira Neto \\
	    \end{tabbing}
    \end{flushleft}
	\vspace{1cm}
	
	\begin{center}
		\vspace{\fill}
			 Junho\\
		 2018
			\end{center}
\end{titlepage}

\begin{titlepage}
	\begin{center}
	
	%\begin{figure}[!ht]
	%\centering
	%\includegraphics[width=2cm]{c:/ufba.jpg}
	%\end{figure}

		\Huge{Universidade Federal de Uberlândia}\\
		\large{Faculdade de Engenharia Mecânica}\\ 
		\large{Laboratório de Mecânica dos Fluidos Computacional}\\ 
\vspace{15pt}
        
        \vspace{85pt}
        
		\textbf{\LARGE{Introdução ao estudo diferencial analítico e à metodologia computacional em
mecânica dos fluidos}}
		%\title{\large{Título}}
	%	\large{Modelo\\
     %   		Validação do modelo clássico}
			
	\end{center}
\vspace{1,5cm}
	
	\begin{flushright}

   \begin{list}{}{
      \setlength{\leftmargin}{4.5cm}
      \setlength{\rightmargin}{0cm}
      \setlength{\labelwidth}{0pt}
      \setlength{\labelsep}{\leftmargin}}

      \item Relatório das atividades executadas no Laboratório de Mecânica dos Fluidos Computacional como requisito para validação do programa de iniciação científica para o curso de engenharia mecânica da Universidade Federal de Uberlândia.
      %Primeiro Relatório de Projeto de Pesquisa apresentado ao Programa XXX do Curso XXXX da Universidade XXX, como requisito parcial para .

      \begin{list}{}{
      \setlength{\leftmargin}{0cm}
      \setlength{\rightmargin}{0cm}
      \setlength{\labelwidth}{0pt}
      \setlength{\labelsep}{\leftmargin}}

			\item Aluno: Ítalo Augusto Magalhães de Ávila \
            \item Prof. orientador: Aristeu da Silveira Neto\

      \end{list}
   \end{list}
\end{flushright}
\vspace{1cm}
\begin{center}
		\vspace{\fill}
		 Junho\\
		 2018
			\end{center}
\end{titlepage}
\newpage
  	   
  	 
  	 \newpage
  	 
  	 \begin{huge}
\textbf{Resumo}
\\

\end{huge}

\noindent
	
	No presente trabalho procura-se desenvolver dinamicamente e termicamente um canal turbulento de Poiseuille unidimensional por meio de uma metodologia semi-analítica. Para tal,  utilizou-se as equações diferenciais da energia térmica, de Navier-Stokes e da continuidade, que foram desenvolvidas numericamente e acopladas em um código final capaz de simular o sistema em questão. Foi analisado o erro com base em DNS (Simulações Numéricas Diretas) a fim de se validar o método. \\
	A partir dos resultados iniciais foi notado que algumas constantes tinham grande influencia sobre o resultado (Prandtl turbulento e constante de cebeci), e se experimentou muda-las de forma a se obter o melhor resultado, com isso criou-se modelos simplificados para os mesmos de forma a se melhorar a acurácia do método.

\newpage

\begin{huge}
	\textbf{Abstract}
	\\
	
\end{huge}

\noindent

In the present work, it was developed, dynamically and thermally, a one-dimensional Poiseuille turbulent channel flow by means of a semi-analytical methodology. For that, we used the differential equations of thermal energy, Navier-Stokes and continuity, which were developed numerically and coupled in a final algorithm capable of simulating the channel. The error based on DNS (Direct Numerical Simulations) was analyzed in order to validate the method. \\
From the initial results it was noticed that some constants had great influence on the results (turbulent Prandtl number and Cebeci number), and if it was tried to change them in order to obtain the best result, with that simplified models were created for them in order to improve the accuracy of the method.

\newpage
     
	 \tableofcontents    
          
  	 \chapter{Introdução}\label{chap_intro}

%\pagenumbering{arabic} % numerar as paginas em algarismos arabicos (ou romanos)
\thispagestyle{empty} %Oculta o numero da primeira pagina do capitulo
\vspace{3ex}

A dinâmica dos fluidos e das estruturas imersas são ramos da Engenharia Mecânica com as quais se estuda o comportamento das interações fluido-estruturais, visando aperfeiçoar vários processos na indústria, assim como entender fenômenos na natureza. O estudo desse tipo de fenômeno pode ser feito de duas maneiras: a experimentação material e a experimentação virtual. Em ambas as frentes é necessária a modelagem física do problema em análise e a montagem de bancadas experimentais que representem a física do problema a ser estudado. Para a modelagem material, faz-se necessária a construção de bancadas e as respectivas instrumentações para permitir a coleta de dados e análise estatística dos resultados obtidos. A modelagem física consiste na avaliação do problema de interesse e determinação de suposições físicas que visam viabilizar a análise. Na segunda maneira, é necessária a modelagem física e matemática do problema de interesse. A modelagem matemática consiste na obtenção de equações diferenciais, integrais e/ou integro-diferenciais que modelam a física assciada e depois utilizar métodos numéricos apropriados para a discretização das equações. 

Os métodos de análise computacional estão em franco crescimento. É importante destacar que o método computacional não substitui o experimental material, mas o complementa. Além disso, esta metodologia é muito versátil, permitindo uma análise minuciosa do problema físico e uma maior flexibilidade em relação às condições físicas. Alguns experimentos podem ser perigosos de se reproduzir em laboratório, ou até mesmo impossíveis. Algumas desvantagens são a necessidade de modelos numéricos adequados e de computadores robustos, conforme o problema que se está analisando.

A interação entre escoamentos e estruturas é um problema complexo e recorrente em aplicações de engenharia. Esse fenômeno pode ser encontrado em aeronaves, motores a jato, dutos, reatores nucleares e químicos, pontes, torres, plataformas \textit{off-shore}, válvulas de compressores, coração, aneurismas, entre outros. A dinâmica dos fluidos computacional, aliada à solução numérica das equações que modelam a movimentação de estruturas, é uma grande aliada na compreensão dos problemas de interação fluido-estrutura. Trata-se de um problema multidisciplinar, visto que envolve, por exemplo, a mecânica dos fluidos, mecânica das estruturas, engenharia de software e a ciência da computação.

Os escoamentos sobre estruturas cilíndricas podem ser a fonte de vibrações induzidas por estruturas turbilhonares. Essas vibrações podem induzir um aumento das forças fluidodinâmicas, ou seja, arrasto e sustentação, levando assim a um aumento dos esforços aplicados sobre as estruturas. Alguns resultados indicam que escoamentos bidimensionais sobre cilindros circulares mudam, por exemplo, o coeficiente de arrasto médio de 1,3 para 2,2. Além disso, as vibrações podem causar nucleação e propagação de trincas na estrutura conduzindo-a a falha em virtude da fadiga. Em alguns casos, o valor RMS do coeficiente de sustentação pode ser alterado de 0,3 para 1,75, dependendo do regime de operação \cite{Chern2014}. Esses são resultados que justificam a preocupação com o processo de interação fluido-estrutura em cilindros. Isso é especialmente importante quando esses cilindros são dutos pelos quais petróleo ou gás natural são transportados, sobre os quais se têm ondas e/ou correntes marítimas atuando. A manutenção desse tipo de duto é cara, visto que podem estar a centenas de metros da superfície. Qualquer falha nessas estruturas pode causar desastres ambientais e grandes prejuízos. Por isso é importante entender como o processo de interação fluido-estrutura atua sobre a dinâmica do duto para prevenir falhas.

A pesquisa apresentada neste projeto é resultado de cooperação entre o Laboratório de Mecânica dos Fluidos (MFLab), o Laboratório de Mecânica de Estruturas Prof. José Eduardo Tannús Reis (LMEst) da Universidade Federal de Uberlândia (UFU) e com o centro de pesquisa (CENPES) da Petróleo Brasileiro S.A. (Petrobras). A pesquisa está sendo feita utilizando a ferramenta computacional MFSim, que está sendo desenvolvido no MFLab. Nessa ferramenta é possível a execução de simulações de escoamentos incompressíveis levando em consideração a movimentação de estruturas. Para isso, o método da Fronteira Imersa é utilizado. Esta metodologia é particularmente adequada para os problemas que envolvem interação fluido-estrutura, pois permite tratar os domínios do fluido e da estrutura de forma independente. As equações que modelam o escoamento são resolvidas em um domínio euleriano fixo e cartesiano, enquanto a superfície do corpo imerso (estrutura) é representada por um conjunto de pontos lagrangeanos. Através dessa técnica, as forças na interface entre a estrutura e o fluido são avaliadas e utilizadas tanto nos códigos associados ao fluido para imposição da condição de contorno de não deslizamento, quanto na rotina estrutural para o cálculo dos deslocamentos e velocidades da estrutura.

A motivação para a realização deste projeto de pesquisa se dá pelo fato que é alto o custo computacional de simulações associadas ao dimensionamento de estruturas imersas. Este problema se torna maior caso a aplicação de processos de otimização seja necessária. Assim, neste projeto é desenvolvida uma ferramenta computacional dedicada à construção de metamodelos dessas estruturas. Ao final do projeto, a análise dinâmica de diferentes estruturas imersas poderá ser realizada com baixo custo computacional. Neste caso, em vez de resolver as equações do movimento associadas ao problema de interação fluido-estrutura, os usuários da ferramenta acessarão um mapa criado com funções e polinômios obtidos a partir de simulações de alto custo realizadas previamente. Desta forma, reduzindo o tempo computacional para alguns segundos de cálculo viabilizando a utilização de mais testes e avaliações no projeto estrutural. 

Neste projeto, serão realizadas atividades voltadas para o desenvolvimento de uma ferramenta computacional usada na construção de metamodelos para análise estrutural. Assim, diferentes análises e procedimentos de otimização poderão ser realizados de forma mais rápida e eficiente no projeto deste tipo de estruturas. Além disso, será realizada uma análise computacional de escoamentos turbulentos tridimensionais sobre um duto com \textit{strakes}. Desta forma, ao final do projeto, a análise de diferentes estruturas submersas será realizada com baixo custo computacional. Neste sentido, será utilizada uma plataforma desenvolvida no MFlab que vem sendo empregada em aplicações de interesse da indústria de óleo e gás. Novos desenvolvimentos nesse código serão necessários. Com esses novos desenvolvimentos a ferramenta será potencializada para a aplicação em pauta. Esta é a importância deste projeto para o setor de petróleo, gás natural, energia e biocombustíveis.

A presente proposta tem por objetivo dar sequência à cooperação com a construção da ferramenta de metamodelagem para estruturas de \textit{risers}. Além disso, será realizada a simulação numérica de escoamentos tridimensionais sobre dutos com \textit{strakes}. As dimensões dos \textit{strakes}, ou seja, a altura e passo da hélice, são parâmetros críticos no projeto destes sistemas. Assim sendo, torna-se interessante avaliar sua eficiência através de simulações computacionais para diferentes condições de operação. São objetivos específicos: 1) Metamodelagem estrutural: a parametrização do metamodelo é dada pelo conjunto de dados estruturais e hidrodinâmicos, mas a metamodelagem constitui apenas a análise estrutural. O carregamento hidrodinâmico é dado de entrada; 2) Geração de sinais de CD e CL através de CFD pelo MFSim sobre os modelos que temos de cilindros fixos com strakes, ensaiados em canal de água corrente: análises de CFD puras; 3) Metamodelagem para levantamento de cargas sobre cilindros com \textit{strakes} sem acoplamento estrutural (metamodelagem apenas de CFD).

Neste contexto, o presente relatório parcial é organizado da seguinte maneira. \textcolor{red}{\textbf{CONCLUIR CONCLUIR CONCLUIR CONCLUIR CONCLUIR CONCLUIR CONCLUIR}}.
  	 
  	 
\chapter{Modelo Físico}

O problema foi definido como um canal plano, com apenas uma dimensão finita no eixo $ y $. Duas paredes de fronteira foram definidas como placas infinitas em condição de não deslizamento e em regime de fluxo térmico constante. O eixo $ z $ foi proposto como similar tanto na velocidade quanto na temperatura, resultando em um domínio plano (Fig.\ref{figure.1}). \\
A corrente foi considerado incompressível e o fluido foi considerado newtoniano. O fluido flui na direção do eixo $ x $. Os números da Reynolds variam de $ 4560 $ a $ 41441 $, resultando em um regime turbulento.

\begin{figure}[h!]
	\centering
	\includegraphics[angle=0, trim={0mm 10mm 0mm 20mm} , clip , scale=0.50]{fotos_formatacao_final/canal1}
	\caption{Definição geométrica do sistema e fronteiras.}
	\label{figure.1}
\end{figure}

A formulação matemática básica para o problema foram as equações de Navier-Stokes e da continuidade que são apresentadas no livro de Cengel \cite{Cengel}, e a equação de transporte de energia térmica, como apresentada no Incropera \cite{Incropera} de Freank. Estas foram as hipóteses feitas ao problema proposto, que serão consideradas no modelo matemático diferencial a seguir.



\chapter{Modelo Matemático Diferencial}
A equação média da continuidade (Eq.\ref{mass}), a equação média de Navier-Stokes em um eixo cartesiano (Eq.\ref{dynamics}) e a equação média do balanço de energia (Eq.\ref{energy permanent}) são apresentadas a diante: 


\begin{equation}\label{mass}
\frac{\partial \overline{u}}{\partial x} = 0.
\end{equation}

\begin{equation}\label{dynamics}
\frac{\partial \overline{u}\overline{v}}{\partial y} = 
- \frac{1}{\rho} \frac{\partial \overline{p}}{\partial x} + \frac{\partial}{\partial y}\left(\nu \frac{\partial \overline{u}}{\partial y} - \overline{u^\prime v^\prime}\right).
\end{equation}


\begin{equation}\label{energy permanent}
\frac{\partial{}}{\partial{x}} \left(\overline{T^\prime u^\prime}\right) + \frac{\partial{}}{\partial{x}}\left(\overline{u}.\overline{T}\right)     + 
\frac{\partial{}}{\partial{y}} \left(\overline{T^\prime v^\prime}\right) + \frac{\partial{}}{\partial{y}}\left(\overline{v}.\overline{T}\right) 
=
{\frac{\partial{}}{\partial{x}}} \left(\alpha {\frac{\partial{\overline{T}}}{\partial{x}}} \right) +
{\frac{\partial{}}{\partial{y}}} \left(\alpha {\frac{\partial{\overline{T}}}{\partial{y}}} \right). 
\end{equation}


A independência quanto ao tempo e o tratamento em valores médios tornam esta uma metodologia de Reynolds Averaged Navier Stokes (RANS).

\section{O regime permanente da temperatura}

O campo de velocidade é completamente desenvolvido no canal, para valores médios (fig.\ref{figure.3}), mas este não é o caso do campo de temperatura, pois um fluxo térmico constante é imposto sobre as paredes.
Mesmo considerando os valores médios, o campo de temperatura de uma correnteza em um canal turbulento não converge naturalmente para um estado permanente unidimensional (fig.\ref{figure.2}). O campo de velocidade está completamente desenvolvido, mas este não é o caso do campo de temperatura, pois um fluxo térmico constante é imposto sobre as paredes.

Em um esforço para simplificar a solução, a configuração térmica foi estudada com uma equação de balanço de energia térmica (\ref{c_h_e}).
\begin{figure}[h!]
	\centering
	\includegraphics[angle=0, scale=0.40]{fotos_formatacao_final/temperatura}
	\caption{Campo de temperatura em regime estatístico permanente sobre um canal.}
	\label{figure.2}
\end{figure}
\begin{figure}[h!]
	\centering
	\includegraphics[angle=0, height=1.4cm , width=12.3cm]{fotos_formatacao_final/velocidade}
	\caption{campo de velocidade em regime completamente desenvolvido ao longo de um canal.}
	\label{figure.3}
\end{figure}


\begin{equation}\label{c_h_e}
q_{conv.} = \dot{m} C_p \Delta T_m,
\end{equation}
\begin{equation}
2q_w b \Delta x = \dot{m} C_p \Delta T_m,
\end{equation}\\



sendo $b$ a profundidade do canal e $T_m$ a temperatura média em uma secção transversal. Então, substituindo $ \dot{m} = u_m 2R b \rho $, e assumindo $ \Delta T_m = \frac{\partial{\left(\overline{T}_m\right)}}{\partial{x}} \Delta x $:
\begin{equation}
2q_w b \Delta x = u_m 2R b \rho  C_p \frac{\partial{\left(\overline{T}_m\right)}}{\partial{x}} \Delta x.
\end{equation}     
\begin{equation}
q_w = u_m R \rho  C_p \frac{\partial{\left(\overline{T}_m\right)}}{\partial{x}} .
\end{equation} 
\begin{equation}\label{c_h_ee}
\frac{\partial{\left(\overline{T}_m\right)}}{\partial{x}} = \frac{q_w}{u_m  R \rho  C_p } .
\end{equation} 

Como todos os termos no lado direito são constantes, a temperatura média teve que variar linearmente na direção do fluxo.
Para entender melhor a temperatura da parede com esse perfil de energia, foi realizado um estudo convectivo do fluxo térmico, que pode ser expresso matematicamente por:
\begin{equation}
q_w = h A \left( T_w(x) - \overline{T}_m(x)\right).
\end{equation}
É válido observar que $h$ é constante visto que este é um escoamento completamente desenvolvido. Assim, é possível escrever:
%\begin{equation}
%T_w(x) - \overline{T}_m(x) = \frac{q_w}{hA}.
%\end{equation}
%\begin{equation}
%\frac{d T_w(x)}{d x} - \frac{d \overline{T}_m(x)}{d x} = \frac{d \frac{q_w}{hA}}{dx}.
%\end{equation}
\begin{equation}
\frac{d T_w(x)}{d x} = \frac{d \overline{T}_m(x)}{d x} = Cte.
\end{equation}	

Com a temperatura nas paredes e o gradiente médio de temperatura definido como linear por essas determinações matemáticas, foi possível estender esse gradiente para todo o domínio considerando as condições de contorno e a simetria do sistema. Assim, um gradiente de temperatura constante foi imposto nas paredes aquecidas, criando uma condição de contorno de fluxo térmico constante, resultando em todo o campo de temperatura para variar linearmente no eixo $x$. A temperatura foi decomposta então na forma $ T^\ast(y) = T(x,y) - T_w(x) $ onde $T_w(x)$ é a temperatura nas paredes, resultando em uma simetria no sentido da corrente, diminuindo o problema a um unidimensional representativo para a variável $T^\ast(y)$. Assim, a expressão foi substituída em (\ref{energy permanent}):



\begin{equation}
\begin{split}
\frac{\partial{}}{\partial{x}} \left(\overline{(T^\ast + T_w)^\prime} \overline{ u^\prime}\right) + \frac{\partial{}}{\partial{x}}\left(\overline{(T^\ast + T_w)} \overline{u}\right)+ 
\frac{\partial{}}{\partial{y}} \left(\overline{(T^\ast + T_w)^\prime} \overline{ v^\prime}\right) + \frac{\partial{}}{\partial{y}}\left(\overline{(T^\ast + T_w)} \overline{v}\right) = \\
{\frac{\partial{}}{\partial{x}}} \left(\alpha {\frac{\partial{\overline{(T^\ast + T_w)}}}{\partial{x}}} \right) +
{\frac{\partial{}}{\partial{y}}} \left(\alpha {\frac{\partial{\overline{(T^\ast + T_w)}}}{\partial{y}}} \right) 
\end{split}
\end{equation}

Então a expressão poderia ser mais desenvolvida algebricamente considerando toda a velocidade média em $y$ e $z$ nulas:

\begin{equation}\label{equation_var}
{\frac{\partial{}}{\partial{y}}} \left(\alpha {\frac{\partial{\overline{T^\ast}}}{\partial{y}}}   
- \left(\overline{ T^{\ast\prime} v^\prime}\right) \right)
= 
\overline{u}\frac{\partial{}}{\partial{x}}\left(\overline{T_w}\right)  
\end{equation}



\section{A hipótese de Boussinesq}

Agora, em uma expressão simplificada unidimensional, o modelo requer um modelo de fechamento para o fluxo turbulento de energia térmica. Assim, a hipótese de Boussinesq foi usada. O termo $\overline{T^{\ast\prime}  v^\prime}$ pode ser modelado com:
\begin{equation}\label{bou}
-\left(\overline{ T^{\ast\prime}  v^\prime}\right) = 
\alpha_t \frac{\partial{\overline{T^\ast}}}{\partial{y}}.
\end{equation}
Assim, a seguinte equação pode ser obtida substituindo na equação principal (\ref{equation_var}):
\\
\begin{equation}
{\frac{\partial{}}{\partial{y}}} \left[(\alpha + \alpha_t)  \frac{\partial \overline{T^\ast}}{\partial y} \right]
= 
\overline{u}\frac{\partial{}}{\partial{x}}\left(\overline{T_w}\right) . 
\end{equation}

\section{O modelo de comprimento de mistura de Prandtl} 

O termo da difusão térmica turbulenta, $\alpha_t$, precisava ser modelado. O conceito clássico do número de Prandtl turbulento é usado no presente trabalho:
\begin{equation}
Pr_t = \frac{\nu_t}{\alpha_t}.
\end{equation} 
O termo $ \nu_t $ precisa ser modelado. O valor clássico do número de Prandtl turbulento $ Pr_t = 0.71 $ foi utilizado.
Com o modelo do comprimento de mistura de Prandtl, é postulado que:
\begin{equation}
\nu_t = {l^2_m} \left| \frac{\partial \overline{u}}{\partial y} \right|.
\end{equation}
O comprimento de mistura introduz um módulo no modelo diferencial, bem como o parâmetro do número de Prandtl turbulento:
\\
\begin{equation}
{\frac{\partial{}}{\partial{y}}} \left( \left( \alpha   
+ \frac{\nu_t}{Pr_t} \right) \frac{\partial \overline{T^\ast}}{\partial y} \right)
= 
\overline{u}\frac{\partial{}}{\partial{x}}\left(\overline{T_w}\right)  .
\end{equation}
\begin{equation}\label{equationquasela}
{\frac{\partial{}}{\partial{y}}} \left( \left( \alpha   
+ \frac{{l^2_m} \left| \frac{\partial \overline{u}}{\partial y} \right|}{Pr_t} \right) \frac{\partial \overline{T^\ast}}{\partial y} \right)
= 
\overline{u}\frac{\partial{}}{\partial{x}}\left(\overline{T_w}\right)  .
\end{equation}
\\

É possível perceber, ao analisar a dinâmica do fluxo, que para valores positivos de $ y $, ver figura \ref{figure.1}, a primeira derivada da velocidade será sempre negativa, pois temos uma velocidade que diminui com o aumento de $ y $. Isso implica em:\\
\begin{equation}
{\frac{\partial{}}{\partial{y}}} \left( \left( \alpha   
- \frac{{l^2_m}}{Pr_t}\frac{\partial \overline{u}}{\partial y} \right) \frac{\partial \overline{T^\ast}}{\partial y} \right)
= 
\overline{u}\frac{\partial{}}{\partial{x}}\left(\overline{T_w}\right)  .
\end{equation}



\section{O comprimento de mistura}

O comprimento de mistura $ l_m $ precisa ser modelado. Pode-se observar que os estudos experimentais de Nikuradse foram utilizados para modelar este parâmetro, conforme segue::
\begin{equation}
L\left(\frac{y}{R}\right) = \frac{l_m}{R} = 0.14 - 0.08 \left(\frac{y}{R}\right)^2 - 0.06\left(\frac{y}{R}\right)^4.
\end{equation}
Para completar ainda mais o modelo, Cebeci e Bradshaw adicionaram a função de amortecimento Van Driest:
\begin{equation}\label{eqution_mixturelength}
L\left(\frac{y}{R}\right)  = \frac{l_m}{R} = \left\{\frac{l_m}{R} = 0.14 - 0.08 \left(\frac{y}{R}\right)^2 - 0.06\left(\frac{y}{R}\right)^4\right\}\left\{  1 - e^{[(\tilde{y} - 1) \frac{Re_\tau}{A}]}\right\},
\end{equation}
Com $A = 26$ como a constante do Cebeci. Assim, teve-se o comprimento de mistura definido por:
\begin{equation}
lm = L R,
\end{equation}
$ L $ sendo uma função no eixo $ y $, dado pela equação (\ref{eqution_mixturelength}). Assim, a equação (\ref{equationquasela}) pode ser escrita como:
\begin{equation}\label{cebeciconstant}
{\frac{\partial{}}{\partial{y}}} \left( \left( \alpha   
- \frac{{L}^2 R ^2}{Pr_t}\frac{\partial \overline{u}}{\partial y} \right) \frac{\partial \overline{T^\ast}}{\partial y} \right)
= 
\overline{u}\frac{\partial{}\left(\overline{T_w}\right)  }{\partial{x}}.
\end{equation}
Para comparar o presente trabalho com modelos da literatura, esta equação foi adimensionalizada utilizando as coordenadas da parede. Foi considerado: $ \tilde{y} = \frac{y . Re_\tau}{R} $, $ \tilde{\overline{u}} = \frac{\overline{u}}{u_\tau} $ , $ \tilde{\overline{T}} = \frac{\overline{T}}{T_\tau} $ , $ \tilde{\overline{T^\ast}} = \frac{\overline{T^\ast}}{T_\tau} $ , $Re_\tau = \frac{u_\tau R}{\nu}$, $Pr_t = \frac{\nu_t}{\alpha_t}$, $Pr = \frac{\nu}{\alpha}$, $T_\tau = \frac{q_w}{\rho C_p u_\tau}$, $\frac{\partial{\left(T_m\right)}}{\partial{x}} = \frac{q_w}{u_m  R \rho  C_p } $ e $\frac{\partial \overline{p}}{\partial x} = - \frac{u_\tau^2 \rho}{R} $. Isso resultou em:
\\
\begin{equation}\label{equationultima}
{\frac{\partial{}}{\partial{\tilde{y}}}} \left( \left( \frac{Re_\tau}{Pr}   
- \frac{{L}^2 Re_\tau ^3}{Pr_t}\frac{\partial \tilde{\overline{u}}}{\partial \tilde{y}} \right) \frac{\partial \tilde{\overline{T^\ast}}}{\partial \tilde{y}} \right)
= 
\frac{\tilde{\overline{u}}}{\tilde{u_m}}.
\end{equation}
É importante notar que há a velocidade na equação (\ref{equationultima}), ou seja, para o desenvolvimento do problema térmico, é necessário o desenvolvimento do perfil de velocidades no canal. Para isso, um modelo RANS previamente feito (Antonialli and Silveira, 2015) foi usado, como segue:
\begin{equation}\label{finalequationvelocity}	
\frac{\partial \tilde{\overline{u}}}{\partial \tilde{y}} = - \frac{2 \tilde{y} \frac{1}{Re_\tau} }{ 1 + \sqrt{ 1 + 4 L ^2 Re_\tau ^2 \tilde{y}}}.
\end{equation}	
Desta forma, tivemos a primeira derivada da velocidade em uma forma exata.


  	 
  	 \chapter{Modelo Matemático Numérico}


Para discretizar uma equação diferencial, um domínio euleriano foi formulado. Para a velocidade, um método de Runge-kutta de quarta ordem foi aplicado, enquanto a temperatura foi organizada em um esquema de diferenças centradas que tinha que ser resolvido implicitamente. O modelo dinâmico é resolvido primeiro, e seu resultado numérico é utilizado na solução do perfil térmico. O centro da célula estava de tal maneira que a parede era colocada no centro da célula e um ponto entre as células era colocado no centro do canal. A convergência dos resultados numéricos é mostrada na figura \ref{sistema}.

\begin{figure}[!h]
	\centering
	\includegraphics[angle=0, trim={10mm 00mm 0mm 0mm}, clip , scale=0.32]{fotos_formatacao_final/convergnciaprimeira}
	\caption{Convergência do modelo para 400 células.}
	\label{sistema}
\end{figure}
  	 
  	 \chapter{Resultados}

A partir dos resultados iniciais, desenvolveram-se mais e mais modificações para melhorá-los.

\section{Resultados preliminares}
Inicialmente, o número de Prandtl turbulento, $Pr_t = 0.71$, foi usado como o da literatura. Os resultados obtidos são mostrados na figura \ref{figuraresultados1} e comparados com DNS de (Kawamura, 2007) e (kasagi et al., 1992).\\
\begin{figure*}[h!] 
	\centering
	%		\begin{subfigure}[t]{0.49\textwidth}
	%		\centering
	%		\includegraphics[angle=0, scale=0.32]{fotos_formatacao_final/Temperature_150_0025_classico}
	%		\caption{Temperature configuration for $Re_\tau = 150$, $Pr = 0.025$, $L2 = 0.13$ }
	%		\end{subfigure}%
	\begin{subfigure}[t]{0.49\textwidth}
		\centering
		\includegraphics[angle=0, scale=0.24]{fotos_formatacao_final/Temperature_150_071_classico}
		\caption{Distribuição de temperatura para $Re_\tau = 150$, $L2 = 1.42$}
	\end{subfigure}
	%		\begin{subfigure}[t]{0.49\textwidth}
	%		\centering
	%		\includegraphics[angle=0, scale=0.32]{fotos_formatacao_final/Temperature_395_0025_classico}
	%		\caption{Temperature configuration for $Re_\tau = 395$, $Pr = 0.025$, $L2 = 0.52$}
	%		\end{subfigure}%
	\begin{subfigure}[t]{0.49\textwidth}
		\centering
		\includegraphics[angle=0, scale=0.24]{fotos_formatacao_final/Temperature_395_071_classico}
		\caption{Distribuição de temperatura para $Re_\tau = 395$, $L2 = 1.55$}
	\end{subfigure}
	%		\centering
	%		\begin{subfigure}[t]{0.49\textwidth}
	%		\centering
	%		\includegraphics[angle=0, scale=0.32]{fotos_formatacao_final/Temperature_395_1_classico}
	%		\caption{Temperature configuration for $Re_\tau = 395$, $Pr = 1.0$, $L2 = 1.89$}
	%		\end{subfigure}%
	%		\begin{subfigure}[t]{0.49\textwidth}
	%		\centering
	%		\includegraphics[angle=0, scale=0.32]{fotos_formatacao_final/Temperature_395_2_classico}
	%		\caption{Temperature configuration for $Re_\tau = 395$, $Pr = 2.0$, $L2 = 2.60$}
	%		\end{subfigure}\\
	%		\begin{subfigure}[t]{0.49\textwidth}
	%		\centering
	%		\includegraphics[angle=0, scale=0.32]{fotos_formatacao_final/Temperature_395_5_classico}
	%		\caption{Temperature configuration for $Re_\tau = 395$, $Pr = 5.0$, $L2 = 3.75$}
	%		\end{subfigure}%
	%		\begin{subfigure}[t]{0.49\textwidth}
	%		\centering
	%		\includegraphics[angle=0, scale=0.32]{fotos_formatacao_final/Temperature_395_7_classico}
	%		\caption{Temperature configuration for $Re_\tau = 395$, $Pr = 7.0$, $L2 = 4.24$}
	%		\end{subfigure}\\
	%	    \centering
	%		\begin{subfigure}[t]{0.49\textwidth}
	%		\centering
	%		\includegraphics[angle=0, scale=0.32]{fotos_formatacao_final/Temperature_395_10_classico}
	%		\caption{Temperature configuration for $Re_\tau = 395$, $Pr = 10.0$, $L2 = 4.55$}
	%		\end{subfigure}%
	%		\begin{subfigure}[t]{0.49\textwidth}
	%		\centering
	%		\includegraphics[angle=0, scale=0.32]{fotos_formatacao_final/Temperature_640_0025_classico}
	%		\caption{Temperature configuration for $Re_\tau = 640$, $Pr = 0.025$, $L2 = 0.84$}
	%		\end{subfigure}\\
	\begin{subfigure}[t]{0.49\textwidth}
		\centering
		\includegraphics[angle=0, scale=0.24]{fotos_formatacao_final/Temperature_640_071_classico}
		\caption{Distribuição de temperatura para $Re_\tau = 640$, $L2 = 1.79$}
	\end{subfigure}%
	\begin{subfigure}[t]{0.49\textwidth}
		\centering
		\includegraphics[angle=0, scale=0.24]{fotos_formatacao_final/Temperature_1000_071_classico}
		\caption{Distribuição de temperatura para $Re_\tau = 1020$, $L2 = 2.04$}
	\end{subfigure}%
	\caption{Distribuição de temperatura para $Pr_t = 0.71$, $A = 26$ e $Pr = 0.71$} 
	\label{figuraresultados1}
\end{figure*}	

Os primeiros resultados não foram satisfatórios. Então, notou-se que o número de Prandtl turbulento teve grande influência no resultado, então o número de Prandtl turbulento do DNS (figura \ref{figure5}) foi utilizado como parâmetro no programa, obtendo-se uma norma L2 de $ 0.19 $ para $ Re_t = 640 $. Assim, se observou que o problema se encontrava com a parametrização do número de Prandtl turbulento. Assim, este virou o foco da pesquisa.

\begin{figure*}[h!]
	\centering
	\includegraphics[angle=0, scale=0.4]{fotos_formatacao_final/DNS_PRt}
	\caption{Prandtl do DNS com o número de Prandtl turbulento de acordo com a coordenada $ y $ do canal.}
	\label{figure5}
\end{figure*}

Assim, iniciou-se o esforço para propor uma parametrização ajustada para o número de Prandtl turbulento.
Neste sentido tentou-se ajustar um valor para o qual o erro foi mínimo comparado com o DNS. Neste sentido, a metodologia do algoritmo de evolução diferencial foi aplicada.



\section{A meta-modelagem, com o algoritmo genético: Evolução Diferencial (DE)}
Utilizou-se um algoritmo que buscou um erro mínimo para a função dada, considerando o número de Prandtl turbulento como variável editável e o menor erro como padrão de interesse.
Foi obtido um número de Prandtl turbulento ideal de $ 0.9$ , para o número de Reynolds turbulento de $ 1020$.
\begin{figure}[!h]
	\centering
	\includegraphics[angle=0, scale=0.31]{fotos_formatacao_final/Genetic_amostra}
	\caption{Iterações do algoritmo genético, com simulações para $Re_\tau = 1020$. Convergência em $Pr_t = 0.9 $.}
\end{figure}

O número de Prandtl turbulento obtido, $Pr_t = 0.9$, foi utilizado nos outros Reynolds turbulento, resultando em:
\begin{figure*}[!h]
	\centering
	\begin{subfigure}[t]{0.5\textwidth}
		\centering
		\includegraphics[angle=0, scale=0.24]{fotos_formatacao_final/Temperature_150_071_Prt0905_A26}
		\caption{Distribuição de temperatura para $Re_\tau = 150$, $L2 = 0.34$}
	\end{subfigure}
	\begin{subfigure}[t]{0.45\textwidth}
		\centering
		\includegraphics[angle=0, scale=0.24]{fotos_formatacao_final/Temperature_395_071_Prt0905_A26}
		\caption{Distribuição de temperatura para $Re_\tau = 395$, $L2 = 0.23$}
	\end{subfigure}
	\begin{subfigure}[t]{0.5\textwidth}
		\centering
		\includegraphics[angle=0, scale=0.24]{fotos_formatacao_final/Temperature_640_071_Prt0905_A26}
		\caption{Distribuição de temperatura para $Re_\tau = 640$, $L2 = 0.19$}
	\end{subfigure}
	\begin{subfigure}[t]{0.45\textwidth}
		\centering
		\includegraphics[angle=0, scale=0.24]{fotos_formatacao_final/Temperature_1000_071_Prt0905_A26}
		\caption{Distribuição de temperatura para $Re_\tau = 1020$, $L2 = 0.14$}
	\end{subfigure}	
	\caption{Resultado de simulações térmicas para $Pr_t = 0.9 $, $A = 26$ e $Pr =0.71$ }
\end{figure*}

Mesmo os resultados sendo muito melhores quando comparados ás figuras \ref{figuraresultados1}, o modelo tinha que ser melhorado. O número de Prandtl turbulento variava com o número de Reynolds turbulento, como visto em (fig.\ref{figure5}), assim, um modelo que contemplasse esse fato tinha de ser proposto. 
Para se obter uma curva para o $Pr_t$ em função de $Re_\tau$, o mesmo algoritmo de otimização teve que ser usado para determinar um número de Prandtl turbulento ideal para cada número de Reynolds turbulento disponível na base de dados de DNS. (Kawamura, 2007) e (kasagi et al., 1992).
Os melhores números de Prandtl turbulentos obtidos foram:

\begin{table}[!h]
	\centering
	\caption{Números de Prandtl turbulentos ideais ajustados para cada número de Reynolds turbulento, com $A = 26$}
	\begin{tabular}{ll}
		\hline
		$Re_\tau$ & $Pr_t$\\
		\hline
		150  &   0.94531\\
		395  &   0.89531\\
		640  &   0.89531\\
		1020 &   0.90000\\ 
		\hline
	\end{tabular}
\end{table}



Executando um ajuste de curva polinomial, um modelo ajustado para o número de Prandtl turbulento como uma função do número de Reynolds turbulento foi desenvolvido:
\begin{equation}
\begin{split}
Pr_t = -4.5604 * 10^{-10} Re_\tau^3 + 9.5690 * 10^{-7} Re_\tau^2 - 6.1715 *10 ^{-4} Re_\tau + 1.0178 
\end{split}
\end{equation}
Os resultados das simulações foram muito precisos, ainda mais do que as simulações com os números de Prandtl turbulentos definidos como os valores médios dos dos dados do DNS. (fig.\ref{figure5}). Estes foram:



\begin{figure*}[!h]
	\centering
	\begin{subfigure}[t]{0.5\textwidth}
		\centering
		\includegraphics[angle=0, scale=0.24]{fotos_formatacao_final/Temperature_150_071_Prt(Ret)_A26}
		\caption{Distribuição de temperatura para $Re_\tau = 150$, $L2 = 0.26$}
	\end{subfigure}
	\begin{subfigure}[t]{0.45\textwidth}
		\centering
		\includegraphics[angle=0, scale=0.24]{fotos_formatacao_final/Temperature_395_071_Prt(Ret)_A26}
		\caption{Distribuição de temperatura para $Re_\tau = 395$, $L2 = 0.22$}
	\end{subfigure}
	\begin{subfigure}[t]{0.5\textwidth}
		\centering
		\includegraphics[angle=0, scale=0.24]{fotos_formatacao_final/Temperature_640_071_Prt(Ret)_A26}
		\caption{Distribuição de temperatura para $Re_\tau = 640$, $L2 = 0.17$}
	\end{subfigure}
	\begin{subfigure}[t]{0.45\textwidth}
		\centering
		\includegraphics[angle=0, scale=0.24]{fotos_formatacao_final/Temperature_1000_071_Prt(Ret)_A26}
		\caption{Distribuição de temperatura para $Re_\tau = 1020$, $L2 = 0.14$}
	\end{subfigure}	
	\caption{Resultados de simulações térmicas para $Pr_\tau(Re_\tau)$, $A = 26$ e $Pr =0.71$ }
\end{figure*}



\newpage

Outras maneiras de reduzir a imprecisão foram pesquisadas. O perfil de velocidade foi uma possibilidade, uma vez que desempenha um papel importante no erro do método. Simulações foram realizadas desenvolvendo apenas essa propriedade física, e houve erro associado, como pode ser visto adiante:

\begin{figure*}[!h]
	\centering
	\begin{subfigure}[t]{0.5\textwidth}
		\centering
		\includegraphics[angle=0, scale=0.24]{fotos_formatacao_final/Temperature_150_Avelocity}
		\caption{Distribuição de velocidade para $Re_\tau = 150$, $L2 = 0.47$}
	\end{subfigure}
	\begin{subfigure}[t]{0.45\textwidth}
		\centering
		\includegraphics[angle=0, scale=0.24]{fotos_formatacao_final/Temperature_395_Avelocity}
		\caption{Distribuição de velocidade para $Re_\tau = 395$, $L2 = 0.17$}
	\end{subfigure}
	\begin{subfigure}[t]{0.5\textwidth}
		\centering
		\includegraphics[angle=0, scale=0.24]{fotos_formatacao_final/Temperature_640_Avelocity}
		\caption{Distribuição de velocidade para $Re_\tau = 640$, $L2 = 0.23$}
	\end{subfigure}
	\begin{subfigure}[t]{0.45\textwidth}
		\centering
		\includegraphics[angle=0, scale=0.24]{fotos_formatacao_final/Temperature_1000_Avelocity}
		\caption{Distribuição de velocidade para $Re_\tau = 1020$, $L2 = 0.23$}
	\end{subfigure}	
	\caption{Resultados de simulações dinâmicas para $A = 26$}
\end{figure*}

Um modelo ajustado foi preparado no presente trabalho para a constante de Cebeci $A$ com o objetivo de reduzir o erro e, por extensão, tornar o método mais preciso. O mesmo algoritmo usado para encontrar os números ideais de Prandtl turbulento foi usado para encontrar uma constante de Cebecis ideal para cada número de Reynolds turbulento. Foi utilizada a velocidade do DNS disponível para calcular a norma L2. A constante de Cebeci "A" foi definida como uma variável editável para o programa, e a norma L2 foi definida como um parâmetro de interesse. Os resultados para as constantes ideais do Cebeci podem ser vistos adiante:


\begin{table}[!h]
	\centering
	\caption{Constantes de Cebeci ideais ajustadas para cada Reynolds turbulento.}
	\begin{tabular}{ll}
		\hline
		$Re_\tau$ & $A$\\
		\hline
		150  &   28.616180\\
		395  &   25.673782\\
		640  &   25.001266\\
		1020 &   25.002136\\ 
		\hline
	\end{tabular}
	\label{tablea}
\end{table}

Dos pontos resultantes do algoritmo de otimização, um modelo de função de Cebeci fora ajustado:
\begin{equation}
A = \frac{Re_\tau ^{0.04510621 * \ln(Re_\tau)} *e ^ {5.27528132} }{Re_\tau ^{0.60941173}}
\end{equation}

Em seguida, foram feitas simulações com a função otimizada para um erro mínimo em relação à velocidade, como pode ser visto nas simulações, houve uma melhora nos resultados:


\begin{figure*}[!h]
	\centering
	\begin{subfigure}[t]{0.5\textwidth}
		\centering
		\includegraphics[angle=0, scale=0.24]{fotos_formatacao_final/Temperature_150_Amodeled}
		\caption{Distribuição de velocidade para $Re_\tau = 150$, $L2 = 0.28$}
	\end{subfigure}
	\begin{subfigure}[t]{0.45\textwidth}
		\centering
		\includegraphics[angle=0, scale=0.24]{fotos_formatacao_final/Temperature_395_Amodeled}
		\caption{Distribuição de velocidade para $Re_\tau = 395$, $L2 = 0.16$}
	\end{subfigure}
	\begin{subfigure}[t]{0.5\textwidth}
		\centering
		\includegraphics[angle=0, scale=0.24]{fotos_formatacao_final/Temperature_640_Amodeled}
		\caption{Distribuição de velocidade para $Re_\tau = 640$, $L2 = 0.14$}
	\end{subfigure}
	\begin{subfigure}[t]{0.45\textwidth}
		\centering
		\includegraphics[angle=0, scale=0.24]{fotos_formatacao_final/Temperature_1000_Amodeled}
		\caption{Distribuição de velocidade para $Re_\tau = 1020$, $L2 = 0.13$}
	\end{subfigure}	
	\caption{Resultados para a velocidade com o valor de Cebeci modelado.}
\end{figure*}

Com a função do Cebeci ajustada, um novo grupo de otimizações foi feito, com a mesma metodologia de evolução diferencial, levando em consideração essa nova formulação para a constante do Cebeci. Esse estudo resultou em um novo conjunto de números ótimos de Prandtl turbulentos para cada número de Reynolds de amostra de DNS, como segue:


\begin{table}[!h]
	\centering
	\caption{Números de Prandtl turbulentos ideais ajustados para cada número de Reynolds turbulento, com A modelado.}
	\begin{tabular}{ll}
		\hline
		$Re_\tau$ & $Pr_t$\\
		\hline
		150  &   0.88594\\
		395  &   0.90156\\
		640  &   0.91094\\
		1020 &   0.91406\\ 
		\hline
	\end{tabular}
\end{table}

Um novo modelo pôde ser proposto para o número de Prandtl Turbulento:
\vspace{-2mm}
\begin{equation}
\begin{split}
Pr_t = 4.5290 * 10^{-12} Re_\tau^3 - 5.73952 * 10^{-8} Re_\tau^2 + 9.397 * 10^{-5} Re_\tau + 0.873117480.
\end{split}
\end{equation}


Com essa parametrização, foi feito um novo conjunto de simulações, com os seguintes resultados:

\begin{figure*}[!h]
	\centering
	\begin{subfigure}[t]{0.5\textwidth}
		\centering
		\includegraphics[angle=0, scale=0.24]{fotos_formatacao_final/Temperature_150_071_Prt(Ret)_Avelocity}
		\caption{Distribuição de temperatura para $Re_\tau = 150$, $L2 = 0.212$}
	\end{subfigure}
	\begin{subfigure}[t]{0.45\textwidth}
		\centering
		\includegraphics[angle=0, scale=0.24]{fotos_formatacao_final/Temperature_395_071_Prt(Ret)_Avelocity}
		\caption{Distribuição de temperatura para $Re_\tau = 395$, $L2 = 0.233$}
	\end{subfigure}
	\begin{subfigure}[t]{0.5\textwidth}
		\centering
		\includegraphics[angle=0, scale=0.24]{fotos_formatacao_final/Temperature_640_071_Prt(Ret)_Avelocity}
		\caption{Distribuição de temperatura para $Re_\tau = 640$, $L2 = 0.205$}
	\end{subfigure}
	\begin{subfigure}[t]{0.45\textwidth}
		\centering
		\includegraphics[angle=0, scale=0.24]{fotos_formatacao_final/Temperature_1000_071_Prt(Ret)_Avelocity}
		\caption{Distribuição de temperatura para $Re_\tau = 1020$, $L2 = 0.175$}
	\end{subfigure}	
	\caption{Resultado para simulações térmicas para $Pr_\tau(Re\tau)$, $A(Re_\tau)$ and $Pr =0.71$ }
\end{figure*}

Os valores do Cebeci que resultaram em um pequeno erro para o algoritmo da velocidade não tiveram o mesmo efeito nos resultados de temperatura. O Cebeci foi modelado para o erro de velocidade mínima, não sendo o melhor para a solução térmica. Algebricamente, a constante de Cebeci aparece duas vezes como se vê nas equações \ref{equationultima} e \ref{finalequationvelocity}. Então, é possível propor dois modelos de constantes de Cebeci. Um para a simulação dinâmica e outro para a simulação de temperatura.    


Outro método de ajuste na evolução diferencial é o ajuste multi objetivo. Tal abordagem foi usada para considerar mais de uma variável simultaneamente para otimização. Este método foi usado para ajustar a constante de Cebeci térmica e o número de Prandtl Turbulento para o menor erro (norma L2) no campo de temperatura resultante para cada amostra de DNS. A função dinâmica de Cebeci foi considerada a anteriormente desenvolvida. Novos valores ideais foram encontrados para o número de Prandtl turbulento e a constante térmica de Cebeci:

\begin{table}[!h]
	\centering
	\caption{Números de Prandtl turbulentos ideais e constantes de Cebeci ajustadas para cada número turbulento de Reynolds, com a abordagem multi objetiva. O Cebeci dinâmico continuou o mesmo da tabela \ref{tablea}. }
	\begin{tabular}{llll}
		\hline
		$Re_\tau$ & $Pr_t$ & $A_d$ & $A_v$\\
		\hline
		150  &   0.72530 & 37.25510 & 28.616180\\
		395  &   0.76821 & 34.24176 & 25.673782\\
		640  &   0.81896 & 31.27627 & 25.001266\\
		1020 &   0.86179 & 28.73726 & 25.002136\\ 
		\hline
	\end{tabular}
\end{table}
Com tais dados numéricos, foram propostos novos modelos para o número de Prandtl Turbulento e a constante térmica de Cebeci:

\begin{equation}
A_t = \frac{Re_\tau ^{0.0395059904287 \ln(Re_\tau)^2 - 0.758759596012 \ln(Re_\tau) +  4.66369525666  } }{e ^{5.67034263}}
\end{equation}

\begin{equation}
\begin{split}
Pr_t = -2.4891 * 10^{-10} Re_\tau^3 +  3.60362 * 10^{-7} Re_\tau^2 + 3.7921 *10 ^{-5} Re_\tau + 0.71234 
\end{split}
\end{equation}
Novas simulações foram desenvolvidas com essa parametrização:

\begin{figure*}[!h]
	\centering
	\begin{subfigure}[t]{0.5\textwidth}
		\centering
		\includegraphics[angle=0, scale=0.24]{fotos_formatacao_final/Temperature_150_071_Genetic2temperature}
		\caption{Distribuição de temperatura para $Re_\tau = 150$, $L2 = 0.091$}
	\end{subfigure}
	\begin{subfigure}[t]{0.45\textwidth}
		\centering
		\includegraphics[angle=0, scale=0.24]{fotos_formatacao_final/Temperature_395_071_Genetic2temperature}
		\caption{Distribuição de temperatura para $Re_\tau = 395$, $L2 = 0.049$}
	\end{subfigure}
	\begin{subfigure}[t]{0.5\textwidth}
		\centering
		\includegraphics[angle=0, scale=0.24]{fotos_formatacao_final/Temperature_640_071_Genetic2temperature}
		\caption{Distribuição de temperatura para $Re_\tau = 640$, $L2 = 0.061$}
	\end{subfigure}
	\begin{subfigure}[t]{0.45\textwidth}
		\centering
		\includegraphics[angle=0, scale=0.24]{fotos_formatacao_final/Temperature_1000_071_Genetic2temperature}
		\caption{Distribuição de temperatura para $Re_\tau = 1020$, $L2 = 0.076$}
	\end{subfigure}	
	\caption{Resultados de simulações térmicas para $Pr_\tau(Re_\tau)$, $A_d(Re_\tau)$, $A_t(Re_\tau) $ e $Pr =0.71$, com ajuste multi objetivo.}
	\vspace{-5mm}
\end{figure*}



\chapter{Discussão}


\section{Resultados}

Com esses métodos, foram estudadas as vantagens e desvantagens de cada um. Uma comparação entre seus erros pode ser vista adiante:\\

\begin{figure}[!h]
	\centering
	\includegraphics[angle=0, trim = {0mm 0mm 0mm 10mm}, scale=0.5]{fotos_formatacao_final/gerais}
	\caption{Comparison between all parameterizations for $Re_\tau$ and $Pr_t$.}
\end{figure}

Todos os modelos desenvolvidos no presente trabalho apresentaram melhores resultados que a parametrização clássica de $ Pr_t = 0.71 $ e $ A = 26 $ no escoamento turbulento de Poiseuille. A correção na constante do Cebeci, apesar de representar um erro menor na velocidade, não resultou em erros menores no perfil de temperatura para todo o domínio. O modelo que obteve os melhores resultados foi o desenvolvido com o algoritmo de evolução diferencial multi-objetivo em que foram considerados dois valores de Cebeci para cada domínio (térmico e dinâmico).

\section{Conclusões}

No presente trabalho foi desenvolvido com sucesso uma metodologia semi-analítica para calcular o perfil de temperatura em um canal com o modelo de comprimento de mistura Prandtl, um modelo clássico de fechamento no estudo da turbulência. As validações com o DNS foram satisfatórias. É importante notar que tais valores do número de Prandtl turbulento e constante de Cebeci foram aplicados para este caso particular, pois as parametrizações foram ajustes computacionais particulares e são simplificações das abstrações físicas que cada problema representa. Mas eles se adaptaram bem a este caso, resultando em precisão e eficiência. O método semi analítico foi bem sucedido no modelo do problema, pois bons resultados foram obtidos quando utilizados os números de Prandtl turbulento do DNS. Quanto aos modelos propostos, é uma ideia o aplicar então em outros problemas, com outras geometrias, como forma de testar sua universalidade.
  	 
  	 \newpage
  	 
  	 \begin{large}
\textbf{Referências Bibliográficas}
\end{large} 
\\
\\
Ferziger, J. H. and Peric, M., 2001. Computational Methods for Fluid Dynamics. \\ \\
Fortuna, A. O, 2012. Técnicas Computacionais para Dinâmica dos Fluidos. \\ \\
LeVeque, R. J., 2002. Finite Volume Method for Hyperbolic Problems. \\ \\
LeVeque, R. J., 2007. Finite Difference Method for Ordinary and Partial Differential
Equations. \\ \\
LeVeque, R. J., 1992. Numerical Methods for Conservation Laws. \\ \\
Maliska, C. R. Transferência de Calor e Mecânica dos Fluidos Computacional:
Fundamentos e Coordenadas Generalizadas, 1st Ed. 250 p. LTC, Rio de Janeiro, 1995. \\ \\
Patankar, S. V., 1980. Numerical heat transfer and fluid flow (series in computational
methods in mechanics as thermal sciences). \\ \\
Pivello, M. R., 2012. Um método front-tracking completamente adaptativo para a
simulação de escoamentos tridimensionais bifásicos. Tese de Doutorado, Universidade
Federal de Uberlândia. \\ \\
Roma, A. M., 1996. A Multilevel self-adaptive version of the immersed boundary
method. Ph.D. Thesis, New York University, 1996. \\ \\
Salari, K. and Knupp, P., 2000. Code Verification by the Method of Manufactured
Solutions. Report, Sandia Corporation, California. \\ \\
Villar, M. M., 2007. Análise numérica detalhada de escoamentos multifásicos
bidimensionais. Tese de Doutorado, Universidade Federal de Uberlândia. \\ \\
White, F. M. Fluid Mechanics. 4. ed. New York, NY, USA: McGraw-Hill, 1998.
(McGraw-Hill Series in Mechanical Engineering). \\

%
%\begin{large}
%\textbf{Nota de responsabilidade}
%\end{large} 
%\\
%\\
%	Os autores são responsáveis por todo o material disponibilizado no presente trabalho.
  	 
  \end{document}