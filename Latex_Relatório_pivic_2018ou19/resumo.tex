\begin{huge}
\textbf{Resumo}
\\

\end{huge}

\noindent
	
	A dinâmica dos fluidos é um assunto de grande interesse acadêmico e industrial. Ele oferece
	muitas oportunidades de otimização em uma variedade de problemas práticos de engenharia.
	A modelagem de propriedades térmicas de escoamentos turbulentos é algo de notável
	complexidade, que pode ser explicado matematicamente pela natureza caótica desse problema
	não-linear, fato muito bem explicado no livro de Strogatz "Nonlinear Dynamics and
	Chaos". O resultado imediato de tal fato é a impossibilidade de uma resposta matemática
	exata para, quase, todas as aplicações reais. A única maneira de coletar resultados, então, é
	por abordagem numérica, com a discretização de espaço e tempo, para um resultado aproximado.
	O problema é que, na maioria dos casos, esses são um problema computacional tão
	grande que se torna inviável. Com isso em mente, os autores do presente trabalho pretendem
	desenvolver uma abordagem meta-modelada semi-exata em um problema térmico de canal
	de Poiseuille turbulento, visando eficiência e precisão quando comparada à solução de DNS,
	a mais respeitada mas mais cara computacionalmente. Tal abordagem resultou em um novo
	modelo ajustado para o número de Prandtl turbulento nos canais de Poiseuille.

\newpage

\begin{huge}
	\textbf{Abstract}
	\\
	
\end{huge}

\noindent

Fluid dynamics is a subject of great academic and industrial interest. It offers
many optimization opportunities in a variety of practical engineering problems.
The modeling of thermal properties of turbulent flows is something of remarkable
complexity, which can be explained mathematically by the non-linear chaotic nature of this problem, 
a fact very well explained in Strogatz's book "Nonlinear Dynamics and
Chaos. "The immediate result of this fact is the impossibility of a mathematical answer
to almost any real application. The only way to collect results, then, is
by a numerical approach, with the discretization of space and time, for an approximated result.
The problem is that, in most cases, these are such a large computational problem that becomes unfeasible. 
With this in mind, the authors of the present paper intend
to develop a semi-analytical meta-modeled approach to a thermal channel problem
of turbulent Poiseuille flow aiming at efficiency and precision when compared to the DNS solution,
the most precise but computationally expensive method. This approach has resulted in a new
model adjusted for the number of turbulent Prandtl in Poiseuille channels.

\newpage