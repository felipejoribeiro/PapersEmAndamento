% LaTeX .tex
% Example for the proceedings of the  25th International Congress of Mechanical Engineering
% COBEM 2019
% October, 20-25, 2019, Uberlândia, MG, Brazil
% Based on the template of the proceedings of COBEM2015 and COBEM2017

\documentclass[10pt,fleqn,a4paper,twoside]{article}
\usepackage{abcm}
\def\shortauthor{Ian Costa Alves, Felipe José Oliveira Ribeiro and Alexandre Zuquete Guarato}
\def\shorttitle{Dynamic Modeling of a Rocket for High Accuracy Trajectory Simulations}
\usepackage{subcaption}
\usepackage{amsmath}
\captionsetup{compatibility=false}
\usepackage{blindtext}
\begin{document}
\fphead
\hspace*{-2.5mm}\begin{tabular}{||p{\textwidth}}
\begin{center}
\vspace{-4mm}
\title{Modelagem dinâmica do voo de um foguete para simulações de trajetória de alta exatidão}
\end{center}
\authors{Ian Costa Alves} \\
\authors{Felipe Jose Oliveira Ribeiro} \\
\authors{Alexandre Zuquete Guarato}\\
\institution{Federal University of Uberlândia (UFU), Av. João Naves de Ávila, 2121, Campus Santa Mônica, Uberlândia, MG } \\
\institution{iancostalves@gmail.com} \\
\institution{feliperibeiro.ufu@gmail.com} \\
\institution{azguarato@ufu.br} \\
\\
\abstract{\textbf{Abstract.} Para o projeto de um veículo aeroespacial, simulações de voo de alta fidelidade são essenciais e podem ser críticas para viabilizar ou invalidar o produto. Para o caso de foguetes e mísseis isso se torna ainda mais crítico, visto que a localização exata do local de pouso é um parâmetro determinante para o planejamento do lançamento. Nesse paper, é discutida uma modelagem dinâmica com seis graus de liberdade para o voo de um foguete, além de um modelo aerodinâmico consistente baseado do Método de Barrowman Extendido para a obtenção das forças e momentos aerodinâmicos para o voo do foguete. Além da modelagem, serão mostrados resultados de simulações de voo com o modelo proposto.}\\
\\
\keywords{\textbf{Keywords:} Aerospace, Rocket, Dynamic simulations, Trajectory simulation, Flight mechanics}\\
\end{tabular}

\section{INTRODUCTION}

O modelismo de mini foguetes é uma tarefa muito complexa. O forte carácter multidisciplinar desta atividade resulta em uma constante busca por referencias nas mais diversas áreas da engenharia. Para uma prática segura deste tipo de projeto grande atenção técnica deve ser investida dês das concepções teóricas no início da missão até as partes práticas finais de construção. Para isso, estudos devem ser feitos de forma a se ter sob controle o maior número de variáveis possível, minimizando as chances de falhas. 

O estudo da dinâmica de um mini foguete tem grande importância, visto que possibilita a avaliação das propriedades cinemáticas e das principais solicitações mecânicas experimentadas pela estrutura durante o voo. Para tal, cria-se um modelo representativo da realidade onde se aplicam as leis de balanço de força e energia. Tais equações diferenciais são então integradas numericamente, resultando na trajetória simulada.

\section{METHODOLOGY}
Para o desenvolvimento da dinâmica do foguete, primeiramente, foi determinado o modelo físico tanto do foguete quanto do espaço onde este estaria imerso.


\subsection{Physical model}
O foguete foi considerado como um corpo rígido de massa constante. O referencial cartesiano também considerou o planeta terra como plano, e a gravidade com módulo e sentido constantes na vertical para baixo durante toda a trajetória do veículo. A geometria do foguete pode ser descrita de acordo com a imagem \ref{geometria_foguete}.

\begin{figure}[h!]
	\centering
	\includegraphics[trim = {0cm 13cm 0cm 0cm}, clip , angle=0, scale=0.30]{imagens/foguete_statera}
	\caption{Geometria externa do modelo foguete.}
	\label{geometria_foguete}
\end{figure}

\subsubsection{Eixos de referencia, Equações cinemáticas e matrizes de transformação}
Para se descrever o sistema foi necessária a criação de três eixos de referencia no sistema. Foram criados um referencial fixo $ O $.

Para o devido tratamento matemático e representação das grandezas em referenciais convenientes, foram desenvolvidas matrizes de conversão entre os referenciais desenvolvidos, com na figura \ref{referenciais}

\begin{equation}
\begin{bmatrix}
x_{11}       & x_{12} & x_{13} & \dots & x_{1n} \\
x_{21}       & x_{22} & x_{23} & \dots & x_{2n} \\
\hdotsfor{5} \\
x_{d1}       & x_{d2} & x_{d3} & \dots & x_{dn}
\end{bmatrix}
=
\begin{bmatrix}
x_{11} & x_{12} & x_{13} & \dots  & x_{1n} \\
x_{21} & x_{22} & x_{23} & \dots  & x_{2n} \\
\vdots & \vdots & \vdots & \ddots & \vdots \\
x_{d1} & x_{d2} & x_{d3} & \dots  & x_{dn}
\end{bmatrix}
\end{equation}


\subsubsection{Equações dinâmicas para um foguete com 6 graus de liberdade}

\subsubsection{Modelo de empuxo}

\subsubsection{Modelo aerodinâmico}
Com as equações dinâmicas e o modelo de empuxo prontos, nos resta somente modelar as forças e momentos provenientes da interação do airframe com o escoamento externo e é isso que será desenvolvido nesta seção.

Um modelo aerodinâmico fiel é fundamental para uma boa simulação de voo de um veículo aeroespacial, ainda mais quando tratamos de aeronaves que facilmente atingem velocidades no regime sônico. Desse modo faz-se necessário um modelo não muito complexo, mas capaz de descrever bem o comportamento aerodinâmico do veículo em uma ampla faixa de velocidades, inclusive em altíssimas velocidades.


\subsection{Mathematical model}

\subsection{Computational model}


\section{ACKNOWLEDGEMENTS}

The authors would like to thank the following professors and institutions: FEMEC (Faculdade de Engenharia Mecânica), UFU (Universidade Federal de Uberlândia) and, especially, EPTA (Equipe de Propulsão e Tecnologia Aeroespacial) for financial support.




\section{REFERENCES} 

\bibliographystyle{abcm}
\renewcommand{\refname}{}
\bibliography{bibfile}

\end{document}
